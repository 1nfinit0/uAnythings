\documentclass[10pt, a4paper]{article}
\usepackage[spanish]{babel}
\usepackage[utf8]{inputenc}
\usepackage{tikz}
\usepackage{titling}
\usepackage{graphicx}
\usepackage{amsmath}
\usepackage{amssymb}
\usepackage{geometry}
\usepackage{multicol}
\usepackage{cancel}
\title{\textbf{Resolución de sistemas de ecuaciones lineales por el método Gauss-Jordan} \\ 
\vspace{0.5cm}
\large \textbf{Para el primer ciclo de introducción a la matemática para ingeniería}}
\author{\textbf{Luis Huatay}\\\\\texttt{noggnzzz@gmail.com}\\\\\texttt{U24218809}}


\vspace{-1cm}

\begin{document}
\newgeometry{top=9cm}
\maketitle
\begin{center}
Resolución de las actividades propuestas en la sesión 1 y 2 de la semana 13.
\end{center}
\restoregeometry

\newpage
\newgeometry{
  top=1.5cm,
  bottom=1.5cm,
  left=1.5cm,
  right=1.5cm
}
\section{Retos - Sesión 1}
\subsection{Resolver utilizando el método de eliminación Gaussiana}
\vspace{-0.5cm}
\begin{align*}
  \left\{
  \begin{array}{rcl}
    2x+3y-z &= &9\\
    3x+4y+2z &= &0\\
    x-6y-5z &= &-9
  \end{array}
  \right.\
\end{align*}
\begin{multicols*}{2}
  \textbullet{ \textbf{Sean las matrices:}}
  \begin{align*}
    A &= \begin{bmatrix}
      2 & 3 & -1\\
      3 & 4 & 2\\
      1 & -6 & -5
    \end{bmatrix}&
    B &= \begin{bmatrix}
      5\\
      1\\
      5
    \end{bmatrix}&
    X &= \begin{bmatrix}
      x\\
      y\\
      z
    \end{bmatrix}
  \end{align*}
  \textit{Donde:}
  \begin{align*}
    A: &\text{ Matriz de coeficientes}\\
    B: &\text{ Matriz de términos independientes}\\
    X: &\text{ Matriz de incógnitas}
  \end{align*}
  \textbullet{ \textbf{Sea la matriz aumentada:}}
  \begin{align*}
    \left[A|B\right] = \left[
      \begin{array}{ccc|c}
        2 & 3 & -1 & 9 \\\\
        3 & 4 & 2 & 0 \\\\
        1 & -6 & -5 & -9 
      \end{array}
    \right]
\end{align*}
  \textbullet{ \textbf{Aplicando operaciones elementales:}}
    \begin{align*}
    \left[A|B\right]&=\left[
      \begin{array}{ccc|c}
        2 & 3 & -1 & 9 \\\\
        3 & 4 & 2 & 0 \\\\
        1 & -6 & -5 & -9
      \end{array}
    \right]
    \begin{array}{r}
      \leftarrow F_3\left(-1\right)+F_1 \\\\
      \leftarrow F_3\left(-3\right)+F_2 \\\\
      \\
    \end{array}
  \end{align*}
  \begin{align*}
    \left[A|B\right]&=\left[
      \begin{array}{ccc|c}
        1 & 9 & 4 & 18 \\\\
        0 & 22 & 17 & 27 \\\\
        1 & -6 & -5 & -9
      \end{array}
    \right]
    \begin{array}{r}
      \leftarrow F_2\left(\frac{-9}{22}\right)+F_1 \\\\
      \\
      \\
      \\
    \end{array}
  \end{align*}
  \begin{align*}
    \left[A|B\right]&=\left[
      \begin{array}{ccc|c}
        1 & 0 & \dfrac{-65}{22} & \dfrac{153}{22} \\\\
        0 & 22 & 17 & 27 \\\\
        0 & -6 & \dfrac{-45}{22} & \dfrac{-351}{22}
      \end{array}
    \right]
    \begin{array}{r}
       \\\\\\
      \leftarrow F_2\left(\frac{1}{22}\right) \\\\
      \\
      \\
    \end{array}
  \end{align*}
  \begin{align*}
    \left[A|B\right]&=\left[
      \begin{array}{ccc|c}
        1 & 0 & \dfrac{-65}{22} & \dfrac{153}{22} \\\\
        0 & 1 & \dfrac{17}{22} & \dfrac{27}{22} \\\\
        0 & -6 & \dfrac{-45}{22} & \dfrac{-351}{22}
      \end{array}
    \right]
    \begin{array}{r}
       \\\\\\
       \\\\\\
      \leftarrow F_2\left(6\right)+F_3 \\\\
    \end{array}
  \end{align*}
  \begin{align*}
    \left[A|B\right]&=\left[
      \begin{array}{ccc|c}
        1 & 0 & \dfrac{-65}{22} & \dfrac{153}{22} \\\\
        0 & 1 & \dfrac{17}{22} & \dfrac{27}{22} \\\\
        0 & 0 & \dfrac{57}{22} & \dfrac{-189}{22}
      \end{array}
    \right]
    \begin{array}{r}
      \\\\\\
       \leftarrow F_3\left(\frac{-17}{57}\right)+F_2
      \\\\
    \end{array}
  \end{align*}
  \begin{align*}
    \left[A|B\right]&=\left[
      \begin{array}{ccc|c}
        1 & 0 & \dfrac{-65}{22} & \dfrac{153}{22} \\\\
        0 & 1 & 0 & \dfrac{72}{19} \\\\
        0 & 0 & \dfrac{57}{22} & \dfrac{-189}{22}
      \end{array}
    \right]
    \begin{array}{r}
      \leftarrow F_3\left(\dfrac{65}{57}\right)+F_1
      \\\\
      \\\\
      \\\\
    \end{array}
  \end{align*}
  \begin{align*}
    \left[A|B\right]&=\left[
      \begin{array}{ccc|c}
        1 & 0 & 0 & \dfrac{-54}{19} \\\\
        0 & 1 & 0 & \dfrac{72}{19} \\\\
        0 & 0 & \dfrac{57}{22} & \dfrac{-189}{22}
      \end{array}
    \right]
    \begin{array}{r}
      \\\\
      \\\\\\
      \leftarrow F_3\left(\dfrac{65}{57}\right)+F_1
    \end{array}
  \end{align*}
  \begin{align*}
    \left[A|B\right]&=\left[
      \begin{array}{ccc|c}
        1 & 0 & 0 & \dfrac{-54}{19} \\\\
        0 & 1 & 0 & \dfrac{72}{19} \\\\
        0 & 0 & 1 & \dfrac{-63}{19} 
      \end{array}
    \right]
  \end{align*}
  \textbullet{ \textbf{Finalmente:}}\\
  \begin{align*}
    B&=\left[
      \begin{array}{c}
        \dfrac{-54}{19}\\\\
        \dfrac{72}{19}\\\\
        \dfrac{-63}{19}
      \end{array}
    \right]
    &=
    &&X&=\left[
      \begin{array}{c}
        x \\\\
        y \\\\
        z 
      \end{array}
    \right]
  \end{align*}
  \begin{align*}
    \therefore \ &C.S. \left(x,y,z\right) = \left\{\left(\dfrac{-54}{19},\dfrac{72}{19},\dfrac{-63}{19}\right)\right\}
  \end{align*}
  \columnseprule=1pt
\end{multicols*}
\newpage
\subsection{Una fábrica posee tres máquinas A, B, C, las cuales trabajan en un día, durante 15, 22 y 23 horas respectivamente. Se producen tres artículos X, Y, Z en estas máquinas, en un día y de la siguiente manera: la unidad X está en A durante 1 hora, en B durante 2 horas y en C durante 1 hora; la unidad Y está en A durante 2 horas, en B durante 2 horas y en C durante 3 horas; la unidad Z está en A durante 1 hora, en B durante 2 horas y en C durante 2 horas. Si las maquinas se usan a máxima capacidad, durante un día, Hallar el número de cada artículo que es posible producir.}
\begin{multicols}{2}
\textbullet{ \textbf{Dada la tabla de composición:}}\\
\begin{align*}
  \begin{tabular}{c|c|c|c|c}
    & x & y & z & Total\\
    \hline
    \textit{A} & 1 & 2 & 1 & 15\\
    \textit{B} & 2 & 2 & 2 & 22\\
    \textit{C} & 1 & 3 & 2 & 23\\
  \end{tabular}
\end{align*}
\textbullet{ \textbf{Sea el sistema de ecuaciones:}}
\begin{align*}
  \left\{
  \begin{array}{rcl}
    x+2y+z &= &15\\
    2x+2y+2z &= &22\\
    x+3y+2z &= &23
  \end{array}
  \right.\
\end{align*}
\textbullet{ \textbf{Entonces las matrices:}}
\begin{align*}
  A &= \begin{bmatrix}
    1 & 2 & 1\\
    2 & 2 & 2\\
    1 & 3 & 2
  \end{bmatrix}&
  B &= \begin{bmatrix}
    15\\
    22\\
    23
  \end{bmatrix}&
  X &= \begin{bmatrix}
    x\\
    y\\
    z
  \end{bmatrix}
\end{align*}
\textit{Donde:}
\begin{align*}
  A: &\text{ Matriz de coeficientes}\\
  B: &\text{ Matriz de términos independientes}\\
  X: &\text{ Matriz de incógnitas}
\end{align*}
\textbullet{ \textbf{Aplicando el método de Gauss-Jordan:}}
\begin{align*}
  \left[A|B\right]&=\left[
    \begin{array}{ccc|c}
      1 & 2 & 1 & 15 \\\\
      2 & 2 & 2 & 22 \\\\
      1 & 3 & 2 & 23
    \end{array}
  \right]
  \begin{array}{r}
    \\\\
    \leftarrow F_1\left(-2\right)+F_2 \\\\
    \leftarrow F_1\left(-1\right)+F_3
  \end{array}
\end{align*}
\begin{align*}
  \left[A|B\right]&=\left[
    \begin{array}{ccc|c}
      1 & 2 & 1 & 15 \\\\
      0 & -2 & 0 & -8 \\\\
      0 & 1 & 1 & 8
    \end{array}
  \right]
  \begin{array}{r}
    \\
    \leftarrow F_2\left(-\frac{1}{2}\right)
    \\\\
  \end{array}
\end{align*}
\begin{align*}
  \left[A|B\right]&=\left[
    \begin{array}{ccc|c}
      1 & 2 & 1 & 15 \\\\
      0 & 1 & 0 & 4 \\\\
      0 & 1 & 1 & 8
    \end{array}
  \right]
  \begin{array}{r}
    \leftarrow F_2\left(-2\right)+F_1
    \\\\
    \\\\\\
  \end{array}
\end{align*}
\begin{align*}
  \left[A|B\right]&=\left[
    \begin{array}{ccc|c}
      1 & 0 & 1 & 7 \\\\
      0 & 1 & 0 & 4 \\\\
      0 & 1 & 1 & 8
    \end{array}
  \right]
  \begin{array}{r}
    \\\\
    \\\\
    \leftarrow F_3\left(-1\right)+F_1
  \end{array}
\end{align*}
\begin{align*}
  \left[A|B\right]&=\left[
    \begin{array}{ccc|c}
      1 & 0 & 1 & 7 \\\\
      0 & 1 & 0 & 4 \\\\
      0 & 0 & 1 & 4
    \end{array}
  \right]
  \begin{array}{r}
    \leftarrow F_3\left(-1\right)+F_1
    \\\\
    \\\\
    \\
  \end{array}
\end{align*}
\columnbreak
\begin{align*}
  \left[A|B\right]&=\left[
    \begin{array}{ccc|c}
      1 & 0 & 0 & 3 \\\\
      0 & 1 & 0 & 4 \\\\
      0 & 0 & 1 & 4
    \end{array}
  \right]
\end{align*}
\textbullet{ \textbf{Finalmente:}}
\begin{align*}
  B&=\left[
    \begin{array}{c}
      3\\\\
      4\\\\
      4
    \end{array}
  \right]
  &=
  &&X&=\left[
    \begin{array}{c}
      x \\\\
      y \\\\
      z 
    \end{array}
  \right]
\end{align*}
\begin{align*}
  \therefore \ &C.S. \left(x,y,z\right) = \left\{\left(3,4,4\right)\right\}
\end{align*}
\columnseprule=1pt
\end{multicols}
\newpage
\subsection{Un nutriólogo desea alimentar a un sujeto con una dieta diaria de tres suplementos de dieta MiniCal, LiquiFast y SlimQuick. Es importante que el sujeto consuma exactamente 500 mg de potasio, 75 gr de proteína y 1150 unidades de vitamina D cada día. MiniCal contiene 50 mg de potasio, 5 g de proteína y 90 unidades de vitamina D; LiquiFast contiene 75 mg de potasio, 10 g de proteína y 100 unidades de vitamina D; SlimQuick contiene 10 mg de potasio, 3 g de proteína y 50 unidades de vitamina D. ¿Cuántas onzas de cada alimento debe consumir cada día para satisfacer exactamente las necesidades de nutrientes?}
\begin{multicols*}{2}
  \columnseprule=1pt
  \textbullet{ \textbf{Dada la tabla de composición:}}
\begin{align*}
  \begin{tabular}{c|c|c|c|c}
    & MiniCal & LiquiFast & SlimQuick & Total\\
    \hline
    \textit{Potasio} & 50 & 75 & 10 & 500\\
    \textit{Proteína} & 5 & 10 & 3 & 75\\
    \textit{Vitamina D} & 90 & 100 & 50 & 1150
  \end{tabular}
\end{align*}
\textbullet{ \textbf{Sea el sistema de ecuaciones:}}
\begin{align*}
  \left\{
  \begin{array}{rcl}
    10x+15y+2z &= &100\\
    5x+10y+3z &= &75\\
    9x+10y+5z &= &115
  \end{array}
  \right.\
\end{align*}
\textbullet{ \textbf{Entonces las matrices:}}
\begin{align*}
  A &= \begin{bmatrix}
    10 & 15 & 2\\
    5 & 10 & 3\\
    9 & 10 & 5
  \end{bmatrix}&
  B &= \begin{bmatrix}
    100\\
    75\\
    115
  \end{bmatrix}&
  X &= \begin{bmatrix}
    x\\
    y\\
    z
  \end{bmatrix}
\end{align*}
\textbullet{ \textbf{Método Gauss-Jordan:}}
\begin{align*}
  \left[A|B\right]&=\left[
    \begin{array}{ccc|c}
      10 & 15 & 2 & 100 \\\\
      5 & 10 & 3 & 75 \\\\
      9 & 10 & 5 & 115
    \end{array}
  \right]
  \begin{array}{r}
    \leftarrow F_3\left(-1\right)+F_1 \\\\\\\\\\
  \end{array}
\end{align*}
\begin{align*}
  \left[A|B\right]&=\left[
    \begin{array}{ccc|c}
      1 & 5 & -3 & -15 \\\\
      5 & 10 & 3 & 75 \\\\
      9 & 10 & 5 & 115
    \end{array}
  \right]
  \begin{array}{r}
    \\\\
    \leftarrow F_1\left(-5\right)+F_2 \\\\
    \leftarrow F_1\left(-9\right)+F_3
  \end{array}
\end{align*}
\begin{align*}
  \left[A|B\right]&=\left[
    \begin{array}{ccc|c}
      1 & 5 & -3 & -15 \\\\
      0 & 5 & 6 & -50 \\\\
      0 & -35 & 32 & 250
    \end{array}
  \right]
  \begin{array}{r}
    \leftarrow F_2\left(-1\right)+F_1
    \\\\\\\\
    \leftarrow F_2\left(7\right)+F_3
  \end{array}
\end{align*}
\begin{align*}
  \left[A|B\right]&=\left[
    \begin{array}{ccc|c}
      1 & 0 & 3 & 35 \\\\
      0 & 5 & 6 & -50 \\\\
      0 & 0 & -10 & -100
    \end{array}
  \right]
  \begin{array}{r}
    \\
    \leftarrow F_3\left(-\frac{3}{5}\right)+F_2
    \\\\
  \end{array}
\end{align*}
\begin{align*}
  \left[A|B\right]&=\left[
    \begin{array}{ccc|c}
      1 & 0 & 3 & 35 \\\\
      0 & 5 & 0 & 10 \\\\
      0 & 0 & -10 & -100
    \end{array}
  \right]
  \begin{array}{r}
    \\\\
    \leftarrow F_2\left(\frac{1}{5}\right)
    \\\\
    \leftarrow F_2\left(-\frac{1}{10}\right)
  \end{array}
\end{align*}
\begin{align*}
  \left[A|B\right]&=\left[
    \begin{array}{ccc|c}
      1 & 0 & 3 & 35 \\\\
      0 & 1 & 0 & 2 \\\\
      0 & 0 & 1 & 10
    \end{array}
  \right]
  \begin{array}{r}
    \leftarrow F_3\left(-3\right)+F_1
    \\\\\\
    \\\\
  \end{array}
\end{align*}
\begin{align*}
  \left[A|B\right]&=\left[
    \begin{array}{ccc|c}
      1 & 0 & 0 & 5 \\\\
      0 & 1 & 0 & 2 \\\\
      0 & 0 & 1 & 10
    \end{array}
  \right]
\end{align*}
\textbullet{ \textbf{Finalmente:}}
\begin{align*}
  B&=\left[
    \begin{array}{c}
      5\\\\
      2\\\\
      10
    \end{array}
  \right]
  &=
  &&X&=\left[
    \begin{array}{c}
      x \\\\
      y \\\\
      z 
    \end{array}
  \right]
\end{align*}
\begin{align*}
  \therefore \ &C.S. \left(x,y,z\right) = \left\{\left(5,2,10\right)\right\}
\end{align*}
\end{multicols*}
\newpage
\subsection{Un ingeniero dispone de 5000 horas hombre de mano de obra para tres proyectos. Los costos por hora hombre del primer, segundo y tercer proyecto son \$16, \$20, \$24 respectivamente y el costo total es de \$ 106000. Si el número horas hombre para el tercer proyecto es igual a la suma de las horas hombre requeridas por los dos primeros proyectos. Calcule el número de horas hombre que puede disponerse en cada proyecto.}
  \textbullet{ \textbf{Dado el sistema de ecuaciones:}}
  \begin{align*}
    \left\{
    \begin{array}{rcl}
      x+y+z &= &5000\\
      16x+20y+24z &= &106000\\
      x+y-z &= &0
    \end{array}
    \right.\
  \end{align*}
  \textbullet{ \textbf{Entonces las matrices:}}
  \begin{align*}
    A &= \begin{bmatrix}
      1 & 1 & 1\\
      4 & 5 & 6\\
      1 & 1 & -1
    \end{bmatrix}&
    B &= \begin{bmatrix}
      5000\\
      26500\\
      0
    \end{bmatrix}&
    X &= \begin{bmatrix}
      x\\
      y\\
      z
    \end{bmatrix}
  \end{align*}
  \textbullet{ \textbf{Método Gauss-Jordan:}}
  \begin{align*}
    \left[A|B\right]&=\left[
      \begin{array}{ccc|c}
        1 & 1 & 1 & 5000 \\\\
        4 & 5 & 6 & 26500 \\\\
        1 & 1 & -1 & 0
      \end{array}
    \right]
    \begin{array}{r}
      \\\\
      \leftarrow F_1\left(-4\right)+F_2\\\\
      \leftarrow F_1\left(-1\right)+F_3
    \end{array}
  \end{align*}
  \begin{align*}
    \left[A|B\right]&=\left[
      \begin{array}{ccc|c}
        1 & 1 & 1 & 5000 \\\\
        0 & 1 & 2 & 6500 \\\\
        0 & 0 & -2 & -5000
      \end{array}
    \right]
    \begin{array}{r}
      \leftarrow F_2\left(-1\right)+F_1
      \\\\
      \leftarrow F_3\left(1\right)+F_1
      \\\\\\
    \end{array}
  \end{align*}
  \begin{align*}
    \left[A|B\right]&=\left[
      \begin{array}{ccc|c}
        1 & 0 & -1 & -1500 \\\\
        0 & 1 & 0 & 1500 \\\\
        0 & 0 & -2 & -5000
      \end{array}
    \right]
    \begin{array}{r}
      \\\\\\\\
      \leftarrow F_3\left(-\frac{1}{2}\right)
    \end{array}
  \end{align*}
  \begin{align*}
    \left[A|B\right]&=\left[
      \begin{array}{ccc|c}
        1 & 0 & -1 & -1500 \\\\
        0 & 1 & 0 & 1500 \\\\
        0 & 0 & 1 & 2500
      \end{array}
    \right]
    \begin{array}{r}
      \leftarrow F_3\left(1\right)+F_1
      \\\\\\\\
      \\
    \end{array}
  \end{align*}
  \begin{align*}
    \left[A|B\right]&=\left[
      \begin{array}{ccc|c}
        1 & 0 & 0 & 1000 \\\\
        0 & 1 & 0 & 1500 \\\\
        0 & 0 & 1 & 2500
      \end{array}
    \right]
  \end{align*}
  \textbullet{ \textbf{Finalmente:}}
  \begin{align*}
    B&=\left[
      \begin{array}{c}
        1000\\\\
        1500\\\\
        2500
      \end{array}
    \right]
    &=
    &&X&=\left[
      \begin{array}{c}
        x \\\\
        y \\\\
        z 
      \end{array}
    \right]
  \end{align*}
  \begin{align*}
    \therefore \ &C.S. \left(x,y,z\right) = \left\{\left(1000,1500,2500\right)\right\}
  \end{align*}
\newpage
\section{Retos - Sesión 2}
\subsection{Resolver por eliminación Gaussiana}
\textbullet{ \textbf{Sea el sistema:}}
\begin{align*}
  \left\{
  \begin{array}{rcl}
    3x+2y-2z &= &4\\
    4x+y-z &= &7\\
    x+4y-4z &= &-2
  \end{array}
  \right.\
\end{align*}
\textbullet{ \textbf{Sean las matrices:}}
  \begin{align*}
    A &= \begin{bmatrix}
      3 & 2 & -2\\
      4 & 1 & -1\\
      1 & 4 & -4
    \end{bmatrix}&
    B &= \begin{bmatrix}
      4\\
      7\\
      -2
    \end{bmatrix}
  \end{align*}
  \textbullet{ \textbf{Eliminación Gaussiana:}}
  \begin{align*}
    \left[A|B\right]&=\left[
      \begin{array}{ccc|c}
        3 & 2 & -2 & 4 \\\\
        4 & 1 & -1 & 7 \\\\
        1 & 4 & -4 & -2
      \end{array}
    \right]
    \begin{array}{r}
      \leftarrow F_3\left(-2\right)+F_1 \\\\
      \leftarrow F_3\left(-4\right)+F_2 \\\\
      \\
    \end{array}
  \end{align*}
  \begin{align*}
    \left[A|B\right]&=\left[
      \begin{array}{ccc|c}
        1 & -6 & 6 & 8 \\\\
        0 & -15 & 15 & 15 \\\\
        1 & 4 & -4 & -2
      \end{array}
    \right]
    \begin{array}{r}
      \\\\\\\\
      \leftarrow F_1\left(-1\right)+F_3
    \end{array}
  \end{align*}
  \begin{align*}
    \left[A|B\right]&=\left[
      \begin{array}{ccc|c}
        1 & -6 & 6 & 8 \\\\
        0 & -15 & 15 & 15 \\\\
        0 & 10 & -10 & -10
      \end{array}
    \right]
    \begin{array}{r}
      \\\\
      \leftarrow F_2\left(-\frac{1}{15}\right)
      \\\\
      \leftarrow F_3\left(\frac{1}{10}\right)
    \end{array}
  \end{align*}
  \begin{align*}
    \left[A|B\right]&=\left[
      \begin{array}{ccc|c}
        1 & -6 & 6 & 8 \\\\
        0 & 1 & -1 & -1 \\\\
        0 & 1 & -1 & -1
      \end{array}
    \right]
    \begin{array}{r}
      \\\\
      \\\\
      \leftarrow F_2\left(-1\right)+F_3
    \end{array}
  \end{align*}
  \begin{align*}
    \left[A|B\right]&=\left[
      \begin{array}{ccc|c}
        1 & -6 & 6 & 8 \\\\
        0 & 1 & -1 & -1 \\\\
        0 & 0 & 0 & 0
      \end{array}
    \right]
  \end{align*}
  \textbullet{ \textbf{Finalmente:}}
  \begin{align*}
    \left\{
    \begin{array}{rcl}
      x-6y+6z &= &8\\
      y &= &t-1\\
      z &= &t
    \end{array}
    \right.\ &&z=t\hspace{1cm};\hspace{1cm}t\in\mathbb{R}
    \end{align*}
    \begin{align*}
      x-6\left(t-1\right)+6t &= 8\\
      x-6t+6+6t &= 8\\
      x &= 2
    \end{align*}
    \begin{align*}
      \therefore \ &C.S. \left(x,y,z\right) = \left\{\left(2,t-1,t\right)\right\}
    \end{align*}
\newpage
\subsection{Resolver por eliminación Gaussiana:}
\textbullet{ \textbf{Sea el sistema:}}
\begin{align*}
  \left\{
  \begin{array}{rcl}
    x-y+z &= &1\\
    -x+y-z &= &-1\\
    2x-2y+2z &= &2
  \end{array}
  \right.\
\end{align*}
\textbullet{ \textbf{Sean las matrices:}}
  \begin{align*}
    A &= \begin{bmatrix}
      1 & -1 & 1\\
      -1 & 1 & -1\\
      2 & -2 & 2
    \end{bmatrix}&
    B &= \begin{bmatrix}
      1\\
      -1\\
      2
    \end{bmatrix}
  \end{align*}
  \textbullet{ \textbf{Eliminación Gaussiana:}}
  \begin{align*}
    \left[A|B\right]&=\left[
      \begin{array}{ccc|c}
        1 & -1 & 1 & 1 \\\\
        -1 & 1 & -1 & -1 \\\\
        2 & -2 & 2 & 2
      \end{array}
    \right]
    \begin{array}{r}
      \\\\
      \leftarrow F_1\left(1\right)+F_2 \\\\
      \leftarrow F_1\left(-2\right)+F_3
    \end{array}
  \end{align*}
  \begin{align*}
    \left[A|B\right]&=\left[
      \begin{array}{ccc|c}
        1 & -1 & 1 & 1 \\\\
        0 & 0 & 0 & 0 \\\\
        0 & 0 & 0 & 0
      \end{array}
    \right]
  \end{align*}
  \textbullet{ \textbf{Finalmente:}}
  \begin{align*}
    \left\{
    \begin{array}{rcl}
      x-y+z &= &1\\
      y &= &r\\
      z &= &t
    \end{array}
    \right.\ &&\hspace{1cm}t,r\in\mathbb{R}
    \end{align*}
    \begin{align*}
      x-r+t &= 1\\
      x &= 1+r-t
    \end{align*}
    \begin{align*}
      \therefore \ &C.S. \left(x,y,z\right) = \left\{\left(r-t+1,r,t\right)\right\}
    \end{align*}
\newpage
\subsection{Una tienda se especializa en la preparación de mezclas de café para conocedores. El dueño desea preparar bolsas de 1 kg que se venderán a 120 dólares, usando café colombiano, brasileño y de Kenia. El costo por kilogramo de estos tres tipos de café es de 140, 70 y 100 dólares respectivamente. Determine la cantidad de cada tipo de café, si el dueño decide usar 1/8 de kilogramo del café brasilero.}
\newpage
\subsection{Una fábrica de muebles construye mesas, sillas y armarios, todas de madera. Cada pieza de mueble requiere tres operaciones: corte de madera, ensamblaje y acabado. Realizar una mesa requiere 30 min de corte, 30 min de ensamblaje y 1 h de acabado; una silla requiere 1 h de corte, 90 min de ensamblaje y 90 min de acabado; un armario requiere 1 h de corte, 1 h de ensamblaje y 2 h de acabado. Los trabajadores de la fábrica pueden realizar 300 h de corte, 400 h de ensamblaje y 590 horas de acabado en cada semana. ¿Cuántas mesas, sillas y armarios deben ser producidos para que se usen todas las horas de trabajo disponibles? ¿O esto es imposible?}
\textbullet\textbf{Dada la tabla de composición:}
\begin{align*}
  \begin{tabular}{c|c|c|c|c}
    & Mesa & Silla & Armario & Total\\
    \hline
    \textit{Corte} & 30 & 60 & 60 & 300\\
    \textit{Ensamblaje} & 30 & 90 & 60 & 400\\
    \textit{Acabado} &60 & 90 & 120 & 590
  \end{tabular}
\end{align*}
\textbullet\textbf{Sea el sistema de ecuaciones:}
\begin{align*}
  \left\{
  \begin{array}{rcl}
    x+2y+2z &= &10\\
    3x+9y+6z &= &40\\
    6x+9y+12z &= &59
  \end{array}
  \right.\
\end{align*}
\textbullet\textbf{Entonces las matrices:}
\begin{align*}
  A &= \begin{bmatrix}
    1 & 2 & 2\\
    3 & 9 & 6\\
    6 & 9 & 12
  \end{bmatrix}&
  B &= \begin{bmatrix}
    10\\
    40\\
    59
  \end{bmatrix}&
  X &= \begin{bmatrix}
    x\\
    y\\
    z
  \end{bmatrix}
\end{align*}
\textbullet\textbf{Por eliminación Gaussiana:}
\begin{align*}
  \left[A|B\right]&=\left[
    \begin{array}{ccc|c}
      1 & 2 & 2 & 10 \\\\
      3 & 9 & 6 & 40 \\\\
      6 & 9 & 12 & 59
    \end{array}
  \right]
  \begin{array}{r}
    \\\\
    \leftarrow F_1\left(-3\right)+F_2 \\\\
    \leftarrow F_1\left(-6\right)+F_3
  \end{array}
\end{align*}
\begin{align*}
  \left[A|B\right]&=\left[
    \begin{array}{ccc|c}
      1 & 2 & 2 & 10 \\\\
      0 & 3 & 0 & 10 \\\\
      0 & -3 & 0 & -1
    \end{array}
  \right]
  \begin{array}{r}
    \\\\
    \\\\
    \leftarrow F_2\left(1\right)+F_3
  \end{array}
\end{align*}
\begin{align*}
  \left[A|B\right]&=\left[
    \begin{array}{ccc|c}
      1 & 2 & 2 & 10 \\\\
      0 & 3 & 0 & 10 \\\\
      0 & 0 & 0 & 9
    \end{array}
  \right]
\end{align*}
\textbullet\textbf{Finalmente:}
\begin{align*}
  \left\{
  \begin{array}{rcl}
    x+y+z &= &10\\
    3y &= &10\\
    0 &= &9
  \end{array}
  \right.\
\end{align*}
\begin{align*}
  \therefore \ &\text{El sistema es inconsistente}\\
  &C.S. \left(x,y,z\right) = \left\{\left(\emptyset\right)\right\}
\end{align*}
\end{document}