\documentclass[10pt, a4paper]{article}
\usepackage[spanish]{babel}
\usepackage[utf8]{inputenc}
\usepackage{tikz}
\usepackage{titling}
\usepackage{graphicx}
\usepackage{amsmath}
\usepackage{amssymb}
\usepackage{geometry}
\usepackage{multicol}
\usepackage{cancel}
\usepackage{pgfplots}
\pgfplotsset{compat=1.18}
\title{\textbf{Funciones} \\ 
\vspace{0.5cm}
\large \textbf{Para el primer ciclo de introducción a la matemática para ingeniería}}
\author{\textbf{Luis Huatay}\\\\\texttt{noggnzzz@gmail.com}\\\\\texttt{U24218809}}


\vspace{-1cm}

\begin{document}
\newgeometry{top=9cm}
\maketitle
\begin{center}
Resolución de las actividades propuestas en la sesión 1 y 2 de la semana 15.
\end{center}
\restoregeometry

\newpage
\newgeometry{
  top=1cm,
  bottom=1cm,
  left=1cm,
  right=1cm
}
\setlength{\columnseprule}{0.5pt}
\begin{multicols*}{2}
  \section{Retos - Sesión 1}
  \vspace{-0.3cm}
  \subsection{\small Determine el dominio y rango de la función:}
  \vspace{-0.8cm}
  \begin{align*}
    f(x)=\left\{ \left(4,n\right),\left(3,4\right),\left(4,n^2 - 12\right),\left(n,1\right),\left(5,2n\right)\right\}
  \end{align*}
  \vspace{-0.2cm}
  \textbullet{ Considerando el concepto de función:}
  \begin{align*}
    \left(4,n\right) = \left(4,n^2 - 12\right) \rightarrow n = n^2 - 12 \rightarrow n^2 - n - 12 = 0
  \end{align*}
  \vspace{-0.8cm}
  \begin{align*}
    n^2 - n - 12 &= 0\\
    n^2 - 4n + 3n - 12 &= 0\\
    n\left(n-4\right)+3\left(n-4\right) &= 0\\
    \left(n+3\right)\left(n-4\right) &= 0\\
    n &= -3,4
  \end{align*}
  \textbullet{ Considerando la función con $n=4$:}
  \begin{align*}
    f(x)=\left\{ \left(4,4\right),\left(3,4\right),\left(\textbf{4,4}\right),\left(\textbf{4,1}\right),\left(5,8\right)\right\}
  \end{align*}
  Se puede observar que para un mismo valor del dominio se tiene diferentes valores del rango lo cual hace que $f$ no sea función.\\
  \textbullet{ Considerando la función con $n=-3$:}
  \vspace{-0.2cm}
  \begin{align*}
    f(x)=\left\{ \left(4,-3\right),\left(3,4\right),\left(4,-3\right),\left(-3,1\right),\left(5,-6\right)\right\}
  \end{align*}
  Se observa que para un valor del dominio solo se tiene un valor del rango lo cual hace que $f$ sea función.
  \begin{align*}
    \therefore Domf: \left\{3,4,5,-3\right\} \hspace{1cm} Ranf: \left\{4,-3,1,-6\right\}
  \end{align*}
  \subsection{\small Dada la función $f$ calcula $E$:}
  \vspace{-0.7cm}
  \begin{align*}
    f(x)=\left\{
      \begin{array}{rcl}
        x-9 &; &0<x<6\\
        x^2-16&; &6\leq x<12\\
        26-2x &; &12 \leq x \leq 20
      \end{array}
      \right.\ E= \dfrac{2f\left(3\right)-f\left(6\right)}{5-\left(f\left(12\right)\right)^3}
  \end{align*}
  \textbullet{ Evaluando:}
  \vspace{-0.7cm}
  \begin{align*}
    &f\left(3\right) = 3-9 = -6\\
    &f\left(6\right) = 6^2-16 = 20\\
    &f\left(12\right) = 26-2\left(12\right) = 2
  \end{align*}
  \textbullet{ Calculando $E$:}
  \vspace{-0.7cm}
  \begin{align*}
    E &= \dfrac{2\left(-6\right)-20}{5-\left(2\right)^3}\\
    E &= \dfrac{-12-20}{5-8}\\
    E &= \dfrac{32}{3}    
  \end{align*}
  \vspace{-0.7cm}
  \subsection{\small Dada la función determine su dominio y rango:}
  \vspace{-0.9cm}
  \begin{align*}
    f=\left\{\left(3, 2m+n\right),\left(5,m-13\right),\left(3,8\right),\left(5,n-6\right),\left(n,9\right)\right\}
  \end{align*}
  \vspace{-0.4cm}
  \textbullet{ Considerando el concepto de función:}
  \begin{align*}
    &\left(3,2m+n\right) = \left(3,8\right) &\rightarrow 2m+n = 8\\
    &\left(5,m-13\right) = \left(5,n-6\right) &\rightarrow m-n = 7
  \end{align*}
  \vspace{-0.4cm}
  \textbullet{ Dado el sistema:}
  \begin{align*}
    \left\{
    \begin{array}{rcl}
      2m+n &= &8\\
      m-n &= &7
    \end{array}
    \right.
  \end{align*}
  \vspace{-0.2cm}
  \textbullet{ Resolviendo el sistema:}
  \begin{align*}
    3m = 15 \rightarrow \textbf{m = 5} \hspace{1cm} \textbf{n = -2}
  \end{align*}
  \vspace{-0.2cm}
  \textbullet{ Considerando la función con $m=5$ y $n=-2$:}
  \begin{align*}
    f=\left\{\left(3,8\right),\left(5,-8\right),\left(3,8\right),\left(5,-8\right),\left(-2,9\right)\right\}\\
    \therefore Domf: \left\{3,5,-2\right\} \hspace{1cm} Ranf: \left\{8,-8,9\right\}
  \end{align*}
  \subsection{\small Dados los conjuntos $A=\left\{-3,6,7,12\right\}$ y $B=\left\{-1,3,5,9\right\}$, determine la relación: $R=\left\{\left(a,b\right)\in A \times B \ / \ 2a-b \leq 10\right\}$}
  \vspace{-0.2cm}
  \textbullet{ Determinando $A \times B$:}
  \vspace{-0.2cm}
  \noindent\begin{align*}
    A \times B &= \left\{
    \begin{array}{cccc}
        \left(-3,-1\right), & \left(-3,3\right), & \left(-3,5\right), & \left(-3,9\right), \\
        \left(6,-1\right), & \left(6,3\right), & \left(6,5\right), & \left(6,9\right), \\
        \left(7,-1\right), & \left(7,3\right), & \left(7,5\right), & \left(7,9\right), \\
        \left(12,-1\right), & \left(12,3\right), & \left(12,5\right), & \left(12,9\right)
    \end{array}
    \right\}
    \end{align*}
    \vspace{-0.2cm}
  \textbullet{ Por regla $2a-b \leq 10$:}
  \begin{align*}
    R &= \left\{
    \begin{array}{llll}
        \left(-3,-1\right), & \left(-3,-3\right), & \left(-3,5\right), & \left(-3,9\right), \\
        \left(6,3\right), & \left(6,5\right), & \left(6,9\right), & \left(7,5\right), \\
        \left(7,9\right)
    \end{array}
    \right\}
\end{align*}
\vspace{-0.9cm}
\section{Retos - Sesión 2}
\vspace{-0.4cm}
  \subsection{\small La gráfica de la función: $y=\dfrac{2}{3}x^2+bx+c$, intersecta al eje $x$ en los puntos $\left(-2,0\right)$ y $\left(5,0\right)$ y al eje $y$ en el punto $\left(0,k\right)$. Halle el valor de: $R=b+c+k$.}
  \textbullet{ Por definición:}
  \vspace{-0.2cm}
  \begin{align*}
    c = \text{Intersección con el eje } y \Rightarrow R=b+2c
  \end{align*}
  \vspace{-0.6cm}
  \begingroup
\setlength{\columnseprule}{0pt} % Desactiva la línea divisora
\begin{multicols*}{2}
    \parbox{\linewidth}{
        \textbullet \ Para $\left(-2,0\right)$:
        \begin{align*}
            0 &= \dfrac{2}{3}\left(-2\right)^2-2b+c\\
            c-2b &= -\dfrac{8}{3}
        \end{align*}
    }
    \columnbreak
    \parbox{\linewidth}{
        \textbullet \ Para $\left(5,0\right)$:
        \begin{align*}
            0 &= \dfrac{2}{3}\left(5\right)^2+5b+c\\
            c+5b &= -\dfrac{50}{3}
        \end{align*}
    }
\end{multicols*}
\endgroup
\textbullet{ Resolviendo el sistema:}
\vspace{-0.1cm}
\begin{align*}
    \left\{
    \begin{array}{rcl}
        c-2b &= &-\dfrac{8}{3}\\
        c+5b &= &-\dfrac{50}{3}
    \end{array}
    \right.
\end{align*}
\vspace{-0.4cm}
\begin{align*}
    7b &= -14 \rightarrow b = -2 \hspace{1cm} c = -\dfrac{50}{3}+10 = -\dfrac{20}{3}
\end{align*}
\textbullet{ Finalmente:}
\vspace{-0.6cm}
\begin{align*}
  R = -2+2\left(\dfrac{-20}{3}\right)\\
  R = -2-\dfrac{40}{3}\\
  R = -\dfrac{46}{3}
\end{align*}
\subsection{\small Halla el dominio, rango y gráfica de la función: $f\left(x\right)=\sqrt[]{6+x-x^2}$}
\textbullet{ Gráfica de la función:}
\begin{center}
  \begin{tikzpicture}[scale=0.8]
    \begin{axis}[
        xmin=-3,xmax=4,
        ymin=-0.5,ymax=5,
        axis x line=middle,
        axis y line=middle,
        xlabel={$x$},
        ylabel={$f(x)$},
        ]
        \addplot[no marks,blue] expression[domain=-2:3,samples=100]{sqrt(6+x-x^2)} 
                    node[pos=0.47,anchor=south west]{$f\left(x\right)=\sqrt[]{6+x-x^2}$}; 
    \end{axis}
\end{tikzpicture}
\end{center}
\vspace{-0.5cm}
\textbullet{ Para el dominio:}
\vspace{-0.5cm}
\begin{align*}
  6+x-x^2 &\geq 0\\
  x^2-x-6 &\leq 0\\
  \left(x-3\right)\left(x+2\right) &\leq 0
\end{align*}
\begin{center}
  \begin{tikzpicture}[scale=0.3]

    \draw (-4,0) -- (5,0); %AXIS
    \foreach \x in {-2,3} {
        \draw (\x,0.5) -- (\x,-0.5) node[below] {$\x$};
    }
    \draw (-4,0.5) -- (-4,-0.5) node[below] {$-\infty$};
    \draw (5,0.5) -- (5,-0.5) node[below] {$\infty$};
    \draw (-2,1) -- (3,1);
    \fill (-2,1) circle (0.25);
    \fill (3,1) circle (0.25);
    \end{tikzpicture}
\end{center}
\vspace{-0.4cm}
\textbullet{ Para el rango:}\\
\vspace{-0.25cm}
\text{Por valor medio del intervalo: $x=\dfrac{1}{2}$}
\begin{align*}
  f\left(\dfrac{1}{2}\right)&=\sqrt[]{6+\dfrac{1}{2}-\left(\dfrac{1}{2}\right)^2}\\
  f\left(\dfrac{1}{2}\right)&= \sqrt[]{6+\dfrac{1}{2}-\dfrac{1}{4}}\\
  f\left(\dfrac{1}{2}\right)&= \sqrt[]{6+\dfrac{1}{4}} = \sqrt[]{\dfrac{25}{4}} = \dfrac{5}{2}
\end{align*}
\vspace{-0.3cm}
\textbullet{ Finalamente:}
\begin{align*}
  Domf: x \in \left[-2,3\right] \hspace{1cm} Ranf: y \in \left[0,\dfrac{5}{2}\right]
\end{align*}
\vspace{-0.7cm}
\subsection{\small Halla el dominio, rango y gráfica de la función: $f\left(x\right)=2x^2+x+6$, con $x \in \left[-3,8\right]$}
\textbullet{ Gráfica de la función:}
\vspace{-0.2cm}
\begin{center}
  \begin{tikzpicture}[scale=0.75]
    \begin{axis}[
        xmin=-4,xmax=9,
        ymin=0,ymax=150,
        axis x line=middle,
        axis y line=middle,
        xlabel={$x$},
        ylabel={$f(x)$},
        ]
        \addplot[no marks,blue] expression[domain=-3:8,samples=100]{2*((x)^2)+x+6} 
                    node[pos=0.9,anchor=south east]{$f\left(x\right)=2x^2+x+6$}; 
    \end{axis}
\end{tikzpicture}
\end{center}
\vspace{-0.5cm}
\textbullet{ Para el rango:}\\
\text{ Considerando una función que dibuja una parábola positiva:}
\text{Vértice: $-\dfrac{b}{2a}\Rightarrow h = -\dfrac{1}{4}$}\\
\vspace{-0.5cm}
\begin{align*}
  k=f\left(-\dfrac{1}{4}\right) &= 2\left(-\dfrac{1}{4}\right)^2-\dfrac{1}{4}+6\\
  k &= 2\left(\dfrac{1}{16}\right)-\dfrac{1}{4}+6\\
  k &= \dfrac{1}{8}-\dfrac{2}{8}+\dfrac{48}{8}\\
  k &= \dfrac{47}{8}
\end{align*}
\text{ Considerando el valor de $x$ más alejado de $h$:}
\begin{align*}
  f\left(8\right) &= 2\left(8\right)^2+8+6\\
  f\left(8\right) &= 2\left(64\right)+8+6\\
  f\left(8\right) &= 128+8+6\\
  f\left(8\right) &= 142
\end{align*}
\textbullet{ Finalmente:}
\begin{align*}
  Domf: x \in \left[-3,8\right] \hspace{1cm} Ranf: y \in \left[\dfrac{47}{8},142\right]
\end{align*}
\vspace{-0.5cm}
\subsection{\small Halla el dominio, rango y gráfica de la función: $f\left(x\right)=\sqrt[]{\dfrac{x^2+2x-15}{x^2-100}}$}
\textbullet{ Gráfica de la función:}
\begin{center}
  \begin{tikzpicture}[scale=0.75]
    \begin{axis}[
        xmin=-20, xmax=20,
        ymin=-0.5, ymax=5,
        axis x line=middle,
        axis y line=middle,
        xlabel={$x$},
        ylabel={$f(x)$},
        domain=-20:20,
        samples=200,
        restrict y to domain=0:5,
        ]
        \addplot[no marks,blue] expression {
          sqrt((x^2 + 2*x - 15) / (x^2 - 100))
        } node[pos=0.65,anchor=south] {$f\left(x\right)=\sqrt{\dfrac{x^2 + 2x - 15}{x^2 - 100}}$};
    \end{axis}
  \end{tikzpicture}
\end{center}
\textbullet{ Para el dominio:}\\
\text{ Cnosiderando la restricción propia de un radical:}
\begin{align*}
  \dfrac{x^2 + 2x - 15}{x^2 - 100} &\geq 0\\
  \dfrac{\left(x+5\right)\left(x-3\right)}{\left(x+10\right)\left(x-10\right)} &\geq 0
\end{align*}
\text{ Puntos críticos:}
  \begin{align*}
    x = 3\\
    x = -5\\
    x = 10\\
    x = -10
  \end{align*}
  Donde: $x = 10 \wedge x = -10 $ son abiertos por formar parte del donominador.\\
  \textbullet{ Finalmente: }
  \begin{center}
    \begin{tikzpicture}[scale=0.3]
      \draw (-14,0) -- (14,0); %AXIS
      \foreach \x in {-10,-5,3,10} {
          \draw (\x,0.5) -- (\x,-0.5) node[below] {$\x$};
      }
      \draw (-14,0.5) -- (-14,-0.5) node[below] {$-\infty$};
      \draw (14,0.5) -- (14,-0.5) node[below] {$\infty$};
      \draw (-14,1) -- (14,1);
      \fill (-5,1) circle (0.25);
      \fill (3,1) circle (0.25);
      \draw[fill=white](-10,1) circle (0.25);
      \draw[fill=white](10,1) circle (0.25);
      \draw[fill=white](-14,1) circle (0.25);
      \draw[fill=white](14,1) circle (0.25);
    \end{tikzpicture}
  \end{center}
\begin{align*}
  \therefore Domf&: x \in \left(-\infty,-10\right) \cup \left[-5,3\right] \cup \left(10,\infty\right)\\
  Ranf&: y \in \mathbb{R}
\end{align*}
\subsection{\small Halla el dominio, rango y gráfica de la función: $f\left(x\right)=x^2+3x-\dfrac{3}{4}$}
\textbullet{ Gráfica de la función:}
\begin{center}
  \begin{tikzpicture}[scale=0.75]
    \begin{axis}[
        xmin=-4,xmax=2,
        ymin=-3.5,ymax=2,
        axis x line=middle,
        axis y line=middle,
        xlabel={$x$},
        ylabel={$f(x)$},
        ]
        \addplot[no marks,blue] expression[domain=-4:2,samples=100]{(x)^2+3*x-3/4} 
                    node[pos=0.1,anchor=west]{$f\left(x\right)=x^2+3x-\dfrac{3}{4}$}; 
    \end{axis}
\end{tikzpicture}
\end{center}
\textbullet{ Para el dominio:}\\
  \text{ Buscando la coordenada x del vértice:}
  \begin{align*}
    f\left(x\right)=x^2+3x-\dfrac{3}{4}\\
    h = -\dfrac{b}{2a} = -\dfrac{3}{2}
  \end{align*}
  \text{ Evaluando en $h$:}
  \begin{align*}
    f\left(-\dfrac{3}{2}\right) &= \left(-\dfrac{3}{2}\right)^2+3\left(-\dfrac{3}{2}\right)-\dfrac{3}{4}\\
    f\left(-\dfrac{3}{2}\right) &= \dfrac{9}{4}-\dfrac{9}{2}-\dfrac{3}{4}\\
    f\left(-\dfrac{3}{2}\right) &= \dfrac{6}{4}-\dfrac{18}{4}\\
    f\left(-\dfrac{3}{2}\right) &= -\dfrac{12}{4}\\
    f\left(-\dfrac{3}{2}\right) &= k = -3
  \end{align*}
  \textbullet{ Finalmente:}\\
  \text{Considerando que no hay restricción para el dominio:}
  \begin{align*}
    \therefore Domf&: x \in \mathbb{R}\\
    Ranf&: y \in \left[-3,\infty\right)
  \end{align*}
\end{multicols*}
\end{document}