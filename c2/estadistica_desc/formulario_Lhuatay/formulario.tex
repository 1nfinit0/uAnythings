\documentclass[11pt, a4paper]{article}
\usepackage[a4paper, margin=1cm]{geometry}
\setlength{\headheight}{17.74934pt}
\addtolength{\topmargin}{-5.74934pt}
\usepackage[spanish]{babel}
\usepackage[utf8]{inputenc}
\usepackage{tikz}
\usepackage{titling}
\usepackage{graphicx}
\usepackage{fancyhdr}
\usepackage{amsmath}
\usepackage{amssymb}
\usepackage{geometry}
\usepackage{multicol}
\usepackage{cancel}
\usepackage{pgfplots}
\pgfplotsset{compat=1.18}
\setlength{\parskip}{1em}
\setlength{\parindent}{0pt}
\begin{document}
\section{Formulario de Estadística}
\begin{multicols}{2}
    Dado el cuadro de datos:
    \begin{center}
        \begin{tabular}{|c|c|c|c|c|c|}
            \hline
            $Clases$ & $x_i$ & $f_i$ & $F_i$ & $h_i$ \% & $H_i$ \% \\
            \hline
            [3.3, 3.8) & 3.55 & 3  & 3  & 7.5 \%  & 7.5 \%  \\

            [3.8, 4.3) & 4.05 & 8  & 11 & 20 \%   & 27.5 \% \\

            [4.3, 4.8) & 4.55 & 14 & 25 & 35 \%   & 62.5 \% \\

            [4.8, 5.3) & 5.05 & 6  & 31 & 15 \%   & 77.5 \% \\

            [5.3, 5.8) & 5.55 & 4  & 35 & 10 \%   & 87.5 \% \\

            [5.8, 6.3) & 6.05 & 5  & 40 & 12.5 \% & 100 \%  \\
            \hline
        \end{tabular}
    \end{center}
    \textbf{Donde:} \\
        $x_i$: Marca de clase (valor del dato). \\
        $f_i$: Frecuencia absoluta. (Cuántas veces se repite tal dato)\\
        $F_i$: Frecuencia acumulada. \\
        $h_i$: Frecuencia relativa. \\
        $H_i$: Frecuencia acumulada relativa. \\
    \textbf{Considerese:} \\
        $n = 40$ \text{ (Total de datos)} \\ato)\\
\end{multicols}
\end{document}