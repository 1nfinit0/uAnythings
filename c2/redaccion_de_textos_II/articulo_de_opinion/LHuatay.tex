\documentclass{article}
\usepackage[a4paper, top=3cm, bottom=2.5cm, left=2.5cm, right=2.5cm]{geometry} % Ajuste de márgenes
\usepackage[spanish]{babel}
\usepackage[utf8]{inputenc}
\usepackage{tabularx}
\usepackage{tikz}
\usepackage{titling}
\usepackage{graphicx}
\usepackage{fancyhdr}
\usepackage{amsmath}
\usepackage{amssymb}
\usepackage{multicol}
\usepackage{cancel}
\usepackage{pgfplots}
\usepackage{hyperref}
\usepackage{bookmark}
\usepackage{enumitem}
\usepackage[dvipsnames]{xcolor} % Add this line to define OliveGreen
\pgfplotsset{compat=1.18}
\usepackage{titlesec} % Para personalizar títulos
\usepackage{tocloft}  % Para mejorar el índice
\usepackage{setspace} % Para controlar el espaciado

% Configuración de Fancyhdr para encabezados y pies de página
\pagestyle{fancy}
\fancyhf{}
\fancyhead[L]{\includegraphics[width=2cm]{assets/logo-utp.png}}
\fancyhead[R]{\textit{Comprensión y Redacción de Textos II}}

\fancyfoot[R]{\thepage} % Número de página alineado a la derecha

% Ajustes de espaciado entre párrafos y márgenes superiores
\setlength{\parskip}{1.5em}
\setlength{\parindent}{0pt}
\setlength{\headheight}{17.26935pt} % Altura del encabezado
\addtolength{\topmargin}{-2.26935pt} % Compensar el aumento de la altura del encabezado
\setlength{\textheight}{23cm}  % Ajusta el alto del texto

% Definición de comandos personalizados
\newcommand{\SubItem}[1]{
    {\setlength\itemindent{15pt} \item[-] #1}
}

\pagenumbering{gobble}

% Título del documento con mejor control de espaciado
\title{
  \includegraphics[width=5cm]{./assets/logo-utp.png} \\
  \vspace{1cm}
  \textbf{Universidad Tecnológica del Perú} \\
  \vspace{2cm}
  \textbf{Versión Final del Artículo de Opinión} \\
  \vspace{1cm}
  \large \textbf{Para el curso de Compresión y Redacción de Textos II.}
}

\author{
  \begin{tabular}{ll}
    Huatay Salcedo, Luis Elías & U24218809 \\
  \end{tabular} \\\\
  \texttt{Sección 12934}
}

% ENVIROMENTS

% \newenvironment{indexPre}{}{}
% \newenvironment{introduccion}{}{}
% \newenvironment{marcoTeorico}{}{}
% \newenvironment{problematica}{}{}
% \newenvironment{objetivoGeneral}{}{}
% \newenvironment{terminosEstadisticos}{}{}
% \newenvironment{recoleccionDeInformacion}{}{}
% \newenvironment{metodologia}{}{}
% \newenvironment{analisisDescriptivo}{}{}

\begin{document}
\maketitle

\begin{center}
  Docente. Mg. Pablo Orlando Rabines Pumayalla
\end{center}

\restoregeometry

\pagenumbering{arabic} 
\setcounter{page}{2}

\begin{center}
  \large\textbf{La Inteligencia Artificial en la Educación Universitaria: Un Recurso Transformador}
\end{center}

La inteligencia artificial (IA) está cambiando el panorama educativo a pasos agigantados. ¡Qué abuso lo que esta tecnología puede lograr en el ámbito universitario! Hoy en día, vemos cómo sobrepasa fronteras y destinos entre tinieblas de incertidumbres, ¿o de realidades? Desde herramientas que personalizan el aprendizaje hasta asistentes virtuales que parecen sacados de una película de ciencia ficción, la IA está dejando su huella en las aulas. Sin embargo, hay quienes, a duras penas, aún ponen el grito en el cielo cuestionando su impacto. ¿Es beneficioso o perjudicial el uso de la IA en la educación universitaria? En mi opinión, no lo puedo entender: ¡es una insensatez no ver el potencial transformador de esta herramienta! De una vez por todas, debemos reconocer que la IA es un recurso extraordinario que puede revolucionar la educación. A continuación, les contaré por qué estoy convencido de que su impacto es, sin lugar a dudas, positivo.

No cabe duda de que la inteligencia artificial está marcando un antes y un después en la educación universitaria. Una de sus mayores ventajas es su capacidad de personalizar el aprendizaje, ¡como tener un tutor personal a tu lado las 24 horas! Herramientas basadas en IA son capaces de identificar las fortalezas y debilidades de cada estudiante, ofreciendo materiales hechos a medida que, sinceramente, dejan en cero a la izquierda a los métodos tradicionales. Esto hace que el proceso educativo sea más efectivo y, lo más importante, significativo. Además, la IA también se ha convertido en una gran aliada para los docentes. ¡Qué falta de entendimiento tienen quienes no valoran esto! Con tareas repetitivas como la corrección de exámenes automatizada, los profesores pueden, por fin, dejar de sentirse atrapados en un ciclo sin fin de trabajo mecánico y concentrarse en guiar y motivar a sus estudiantes. Según informes recientes, estas tecnologías han reducido considerablemente el tiempo de calificación en las universidades, beneficiando tanto a docentes como a alumnos. Si bien algunos creen que esto deshumaniza la enseñanza, considero a ello como una insensata posición. La IA, en realidad, nos permite cortar por lo sano con lo tedioso y enfocarnos en lo esencial.

Por otro lado, el impacto de la IA en la accesibilidad y la equidad es simplemente alucinante. ¡Es como si la universidad llegara a la palma de tu mano! Gracias a herramientas como asistentes virtuales y plataformas en línea, los estudiantes pueden acceder al conocimiento sin importar dónde estén o qué obstáculos enfrenten. Esto es especialmente valioso para quienes llevan las de perder debido a barreras económicas o geográficas. Además, tecnologías como los lectores de texto o sistemas de transcripción en tiempo real han demostrado ser un salvavidas para los estudiantes con discapacidades, abriendo caminos donde antes solo había muros. ¿Cómo no abrazar esta tecnología con entusiasmo? ¡Basta ya de buscarle el pelo al huevo! Aunque algunos argumentan que estas herramientas podrían generar dependencia, yo creo firmemente que la IA es un puente que conecta a los estudiantes con un futuro lleno de posibilidades, permitiéndoles alcanzar su máximo potencial.

En conclusión, la inteligencia artificial se está consolidando como un aliado invaluable en la educación universitaria. ¡Qué irónico que algunos sigan resistiéndose a algo tan beneficioso! Desde la personalización del aprendizaje hasta la mejora en la accesibilidad, la IA está transformando la educación en un sistema más eficiente, inclusivo y, sobre todo, humano. Claro, surgen dudas, y no faltará quien se pase de la raya con críticas injustificadas, pero los beneficios superan con creces cualquier preocupación. Es hora de reconocer que la IA no solo facilita el trabajo de docentes y estudiantes, sino que también nos lleva un paso más allá hacia una nueva era educativa. De una vez por todas, ¡abracemos este cambio y dejemos que la tecnología y la humanidad vayan de la mano hacia el éxito académico!

\end{document}