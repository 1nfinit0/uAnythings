\section{Comentarios}

  \subsection{Cobertura agrícola permanente}
  
    El cálulo para la obtención de la cobertura agrícola permanente pretende ser una correcta forma de poder tomar muestras de áreas que se consideren como tal para el desarrollo de los cálculos de de daño y su correlación con la TSM. Debido a la ténica de la teledetección es posible que no se tome el total del área que se considera cobertura agrícola permanente, pero si la suficiente para poder considerarse una muestra aceptable.

  \subsection{Registro histórico del NDVI}

    La comparación entre los valores NDVI de la cobertura agrícola permanente y la TSM, son información del mes seleccionado comparado con la TSM de dos meses anteriores al mes escogido. Si bien, esto permite poder entender el comportamiento del NDVI frente a la anomalía de temperatura, es importante entender que los cambios significativos refieren a los meses de mayor anomalía (diciembre - marzo) y no a los dos meses anteriores como se hace en el algoritmo. El enfoque de este algoritmo es poder entender el comportamiento del NDVI frente a la anomalía de temperatura 