\vspace*{\fill}

\section{Introducción}

  Cada cierto tiempo ocurren fenómenos naturales que afectan la cobertura agrícola del suelo, fenómenos como el El Niño o La Niña, que surgen debido a la variación anómala de la temperatura superficial del mar (TSM) en el océano Pacífico ecuatorial, lo que provoca cambios en el comportamiento usual del ciclo hídrico produciendo daños en dicha cobertura agrícola. En el caso del Perú, estos fenómenos son de gran importancia debido a la alta dependencia del país en la agricultura y la pesca, lo que hace necesario contar con herramientas que permitan monitorear y analizar estos fenómenos.

  En este contexto, el presente proyecto tiene como objetivo desarrollar un aplicativo que permita monitorear los efectos de la variación en la temperatura superficial del mar en el Índice de Vegetación de Diferencia Normalizada (NDVI) así como con información del índice costero el niño (ICEN) y su correlación. Para ello utilizando técnicas de teledetección, procesamiento de datos, análisis de datos, herramientas estadíticas, entre otros, lo que permite el monitoreo actualizado, la comprensión histórica y el análisis de vulnerabilidad ante estos cambios.

  El aplicativo se desarrolla y despliega en la plataforma Google Earth Engine, que permite el procesamiento y análisis de imágenes satelitales a gran escala y de uso libre.

\vspace*{\fill}