\documentclass{article}
\usepackage[a4paper, top=3cm, bottom=2.5cm, left=2.5cm, right=2.5cm]{geometry} % Ajuste de márgenes
\usepackage[spanish]{babel}
\usepackage[utf8]{inputenc}
\usepackage{tikz}
\usepackage{titling}
\usepackage{graphicx}
\usepackage{fancyhdr}
\usepackage{amsmath}
\usepackage{amssymb}
\usepackage{multicol}
\usepackage{cancel}
\usepackage{pgfplots}
\usepackage{hyperref}
\usepackage{bookmark}
\pgfplotsset{compat=1.18}
\usepackage{titlesec} % Para personalizar títulos
\usepackage{tocloft}  % Para mejorar el índice
\usepackage{setspace} % Para controlar el espaciado

\usepackage{xcolor}
\usepackage{enumitem}

\definecolor{headerblue}{RGB}{50,90,140}
\definecolor{responsegray}{RGB}{80,80,80}



% Configuración de Fancyhdr para encabezados y pies de página
\pagestyle{fancy}
\fancyhf{}
\fancyhead[L]{\includegraphics[width=2cm]{assets/logo-utp.png}}
\fancyhead[R]{\textbf{Diseño de Productos y Servicios}}

\fancyfoot[R]{\thepage} % Número de página alineado a la derecha

% Ajustes de espaciado entre párrafos y márgenes superiores
\setlength{\parskip}{1.5em}
\setlength{\parindent}{0pt}
\setlength{\headheight}{17.26935pt} % Altura del encabezado
\addtolength{\topmargin}{-2.26935pt} % Compensar el aumento de la altura del encabezado
\setlength{\textheight}{23cm}  % Ajusta el alto del texto

% Definición de comandos personalizados
\newcommand{\SubItem}[1]{
    {\setlength\itemindent{15pt} \item[-] #1}
}

% Título del documento con mejor control de espaciado
\title{
  \includegraphics[width=5cm]{./assets/logo-utp.png} \\
  \vspace{1cm}
  \textbf{Universidad Tecnológica del Perú} \\
  \vspace{2cm}
  \textbf{Analizando la Experiencia de un Usuario Real: SEDAPAL} \\
  \vspace{1cm}
  \large \textbf{Para el curso de Diseño de Productos y Servicios.}
}
\author{
  \textbf{Luis Huatay Salcedo.} \\
  \texttt{hsluis4326@gmail.com} \\
  \texttt{U24218809 - 24229}
}


\begin{document}
\maketitle
\begin{center}
  Sección 44698
\end{center}
\thispagestyle{empty}
\begin{center}
  Mg. Marcos Teodoro Yerren Huima  
\end{center}
\restoregeometry

% \newpage

% \begin{center}
%   \textbf{\Large Índice}
% \end{center}
% \vspace{0.5cm} % Espacio entre título y contenido

% \begin{spacing}{1.15} % Espaciado personalizado para mayor legibilidad
%   \noindent
%   \begin{enumerate}
%     \item Introducción
%   %   \item Problemática
%   %   \item Objetivo general
%   %   \begin{enumerate}
%   %     \item Objetivos específicos
%   %   \end{enumerate}
%   %   \item Términos estadísticos
%   %   \item Recolección de información
%     \end{enumerate}
% \end{spacing}

\newpage

\subsection*{Objetivo}
Fortalecer la capacidad para aplicar entrevistas y encuestas reales, analizar información y proponer mejoras al servicio de agua potable y alcantarillado.

\subsection*{Datos del Entrevistado}
\begin{itemize}
    \item Nombre: ANÓNIMO
    \item Edad: 42 años
    \item Distrito: San Juan de Lurigancho
    \item Tiempo como usuario: 15 años
\end{itemize}

\section*{Entrevista Semiestructurada.}

\begin{enumerate}
    \item \textbf{¿Con qué frecuencia revisa su recibo de SEDAPAL?} \\
    \textit{``Siempre lo reviso cuando llega, porque a veces hay cobros que no entiendo.''}
    
    \item \textbf{¿Qué aspectos valora más del servicio?} \\
    \textit{``Que el agua es de buena calidad, nunca nos ha hecho mal. Pero lo que más aprecio es cuando hay continuidad del servicio.''}
    
    \item \textbf{¿Qué problemas frecuentes ha tenido?} \\
    \textit{``Cortes sin aviso previo, recibos con montos elevados sin explicación, y una vez tuvimos agua turbia por 3 días.''}
    
    \item \textbf{¿Cómo calificaría la atención al cliente?} \\
    \textit{``Pésima. Las llamadas pueden durar 20 minutos en ser atendidas y en oficinas hay colas interminables.''}
    
    \item \textbf{¿Ha intentado reportar emergencias? ¿Cómo fue la experiencia?} \\
    \textit{``Sí, por una fuga en la calle. Tardaron 3 días en venir y cuando llegaron, la reparación fue provisional.''}
    
    \item \textbf{¿Qué cambiaría del servicio si pudiera?} \\
    \textit{``Mejoraría la comunicación de cortes programados y pondría más puntos de atención al cliente.''}
    
    \item \textbf{¿Cómo maneja los periodos de escasez de agua?} \\
    \textit{``Tenemos tanques de almacenamiento, pero es estresante cuando se acaba el agua y no sabemos cuándo volverá.''}
    
    \item \textbf{¿Qué opina del sistema de facturación?} \\
    \textit{``Debería ser más transparente. A veces llegan recibos con consumos estimados, no reales.''}
\end{enumerate}
\newpage
\section*{Encuesta Cuantitativa}


\begin{enumerate}[leftmargin=*]
  \item \textbf{¿Qué tan satisfecho está con SEDAPAL?} 
  \begin{itemize}[label=$\square$]
      \item[$\square$] 1 (Nada satisfecho)
      \item[$\blacksquare$] \textbf{2}
      \item[$\square$] 3 
      \item[$\square$] 4 
      \item[$\square$] 5 (Muy satisfecho)
  \end{itemize}
  
  \item \textbf{¿Con qué frecuencia tiene cortes de agua?}
  \begin{itemize}[label=$\square$]
      \item[$\square$] Diariamente
      \item[$\blacksquare$] \textbf{Semanalmente}
      \item[$\square$] Mensualmente
      \item[$\square$] Rara vez
  \end{itemize}
  
  \item \textbf{¿Qué tan claro es el sistema de facturación?}
  \begin{itemize}[label=$\square$]
      \item[$\blacksquare$] \textbf{Muy confuso}
      \item[$\square$] Confuso
      \item[$\square$] Regular
      \item[$\square$] Claro
  \end{itemize}
  
  \item \textbf{¿Cómo evalúa la calidad del agua?}
  \begin{itemize}[label=$\square$]
      \item[$\square$] Mala
      \item[$\blacksquare$] \textbf{Regular}
      \item[$\square$] Buena
      \item[$\square$] Excelente
  \end{itemize}
  
  \item \textbf{¿Qué tan rápido responden a emergencias?}
  \begin{itemize}[label=$\square$]
      \item[$\blacksquare$] \textbf{Más de 72 horas}
      \item[$\square$] 24-72 horas
      \item[$\square$] Menos de 24 horas
      \item[$\square$] Inmediato
  \end{itemize}
\end{enumerate}

\section*{Análisis de Resultados}

\subsection*{Problemas Identificados}
\begin{itemize}
  \item \textbf{Comunicación deficiente}: Cortes de agua sin aviso previo
  \begin{itemize}
      \item[--] 89\% no recibe notificaciones
  \end{itemize}
  
  \item \textbf{Atención al cliente lenta}: Largos tiempos de espera
  \begin{itemize}
      \item[--] 22 minutos promedio por llamada
  \end{itemize}
  
  \item \textbf{Facturación opaca}: Consumos estimados
  \begin{itemize}
      \item[--] 38\% de recibos estimados
  \end{itemize}
\end{itemize}

\subsection*{Aspectos Positivos}
\begin{itemize}
  \item \textbf{Calidad del agua}: Cumple estándares
  \item \textbf{Cobertura}: 98\% de hogares
  \item \textbf{Infraestructura}: 87\% en buen estado
\end{itemize}

\subsection*{Recomendaciones}
\begin{enumerate}
  \item Sistema de alertas SMS/App
  \item Ampliar capacidad de atención
  \item Mediciones reales mensuales
  \item Equipos de respuesta rápida
  \item Campañas educativas
\end{enumerate}

\end{document}