\documentclass[a4paper,10pt]{article}
\usepackage[utf8]{inputenc}
\usepackage[spanish]{babel}
\usepackage{tikz}
\usetikzlibrary{positioning, shapes}
\usepackage{geometry}
\geometry{margin=1.5cm}

\title{Diseño y Presentación de Modelos de Negocio Innovadores}
\author{Trabajo Práctico}
\date{\today}

\begin{document}
\maketitle

\section*{Ejercicio 1: Creación de un Modelo de Negocio con Business Model Canvas}

\textbf{Descripción:} Cada equipo debe diseñar un modelo de negocio innovador utilizando el Business Model Canvas.

\textbf{Instrucciones:}
\begin{enumerate}
    \item Seleccionar una idea de negocio innovadora o tomar una empresa existente y rediseñar su modelo de negocio.
    \item Completar el Business Model Canvas detallando los 9 bloques: Segmentos de clientes, Propuesta de valor, Canales, Relaciones con clientes, Fuentes de ingresos, Recursos clave, Actividades clave, Socios clave, Estructura de costos.
    \item Crear un lienzo visual en herramientas como Miro, Canvanizer o Figma.
    \item Presentar el modelo de negocio en 3 minutos.
\end{enumerate}

\vspace{0.5cm}

\subsection{Empresa:}
\textbf{Descripción:} Una empresa de entrega de alimentos saludables a domicilio que utiliza tecnología para personalizar las opciones según las preferencias del cliente.

\begin{center}
\textbf{Business Model Canvas (Vista Compacta)}
\end{center}

\begin{center}
  \begin{tikzpicture}[scale=0.85, every node/.style={font=\footnotesize, align=center}]
  % Definir tamaño (aumentar altura)
  \def\w{2.5}
  \def\h{2.2} % altura aumentada

  % Socios Clave (ajustado)
  \draw (0,0) rectangle (\w,\h*2.5);
  \node[align=center, text width=2.8cm] at (0.5*\w,1.1*\h) {\textbf{Socios Clave}\\
    \footnotesize{
    • Proveedores de alimentos\\
    • Empresas logísticas\\
    • Desarrolladores tecnológicos\\
    • Nutricionistas\\
    • Aliados comerciales}};

  % Actividades Clave (optimizado)
  \draw (\w,0) rectangle (2.2*\w,\h*2.5);
  \node[align=center, text width=2.8cm] at (1.6*\w,1.1*\h) {\textbf{Actividades Clave}\\
    \footnotesize{
    • Selección de ingredientes\\
    • Preparación de comidas\\
    • Desarrollo de plataforma\\
    • Personalización\\
    • Logística\\
    • Atención al cliente}};

  % Propuesta de Valor (más compacto)
  \draw (2.2*\w,0) rectangle (4.6*\w,\h*2.5);
  \node[align=center, text width=5.5cm] at (3.4*\w,1.1*\h) {\textbf{Propuesta de Valor}\\
    \footnotesize{
    • Comidas saludables personalizadas\\
    • Entregas a domicilio\\
    • Adaptación nutricional\\
    • Tecnología integrada\\
    • Conveniencia y ahorro}};

  % Relación con Clientes (más ajustado)
  \draw (4.6*\w,0) rectangle (6*\w,\h*2.5);
  \node[align=center, text width=2.8cm] at (5.3*\w,1.1*\h) {\textbf{Relación con Clientes}\\
    \footnotesize{
    • Atención personalizada\\
    • Soporte digital\\
    • Recomendaciones\\
    • Programas de fidelidad}};

  % Segmentos de Clientes (optimizado)
  \draw (6*\w,0) rectangle (8.5*\w,\h*2.5);
  \node[align=center, text width=5cm] at (7.25*\w,1.1*\h) {\textbf{Segmentos de Clientes}\\
    \footnotesize{
    • Personas saludables\\
    • Profesionales ocupados\\
    • Deportistas\\
    • Dietas especiales\\
    • Familias urbanas}};

  % Estructura de Costos (más compacto)
  \draw (0,-\h) rectangle (4.25*\w,0);
  \node[align=center, text width=9cm] at (2.125*\w,-0.5*\h) {\textbf{Estructura de Costos}\\
    \footnotesize{
    • Ingredientes • Salarios • Plataforma • Logística • Marketing • Soporte}};

  % Fuentes de Ingresos (optimizado)
  \draw (4.25*\w,-\h) rectangle (8.5*\w,0);
  \node[align=center, text width=9cm] at (6.375*\w,-0.5*\h) {\textbf{Fuentes de Ingresos}\\
    \footnotesize{
    • Ventas directas • Suscripciones • Entregas • Alianzas • Servicios premium}};
  \end{tikzpicture}
\end{center}

\subsection{Validación}
Para validar el modelo de negocio, se pueden realizar encuestas a 50 clientes potenciales. Los posibles resultados mostrarían que el 90\% de los encuestados está interesado en recibir comidas saludables personalizadas a domicilio.
\newpage
\section*{Ejercicio 2: Creación de un Lean Canvas para una Startup}
\textbf{Descripción:} Cada equipo debe diseñar un Lean Canvas para una startup, identificando problemas clave, soluciones y modelo de ingresos.

\textbf{Instrucciones:}
\begin{enumerate}
    \item Elegir una idea de negocio disruptiva enfocada en una necesidad no resuelta.
    \item Completar el Lean Canvas con los siguientes elementos: Problema, Segmentos de clientes, Propuesta de valor única, Solución, Canales, Estructura de costos, Fuentes de ingresos, Métricas clave, Ventaja injusta.
    \item Validar el problema y la solución mediante encuestas o entrevistas a clientes potenciales.
    \item Presentar en clase el Lean Canvas en un pitch de 3 minutos.
\end{enumerate}

\vspace{0.5cm}

\subsection{Startup}

\textbf{Descripción:} Una plataforma de aprendizaje online personalizada que utiliza IA para adaptar el contenido a las necesidades del usuario.

\begin{center}
\textbf{Lean Canvas (Vista Compacta)}
\end{center}
\begin{center}
\begin{tikzpicture}[scale=1.1, every node/.style={font=\small, align=center}]
  % Definir tamaño más grande
    \def\w{4.2}
    \def\h{2.3}

    % Fila superior (3 bloques)
    \draw (0,0) rectangle (\w,\h);
    \node[text width=3.8cm] at (2.1,1.15) {\textbf{Problema}\\Dificultad para encontrar contenido educativo personalizado; baja motivación y abandono en cursos online.};

    \draw (\w,0) rectangle (\w*2,\h);
    \node[text width=3.8cm] at (\w*1.5,1.15) {\textbf{Segmentos de Clientes}\\Estudiantes universitarios, profesionales en formación continua, empresas que capacitan empleados.};

    \draw (\w*2,0) rectangle (\w*3,\h);
    \node[text width=3.8cm] at (\w*2.5,1.15) {\textbf{Propuesta de Valor Única}\\Plataforma que adapta el contenido y ritmo de aprendizaje usando IA, mejorando resultados y motivación.};

    % Fila media (3 bloques)
    \draw (0,\h) rectangle (\w,\h*2);
    \node[text width=3.8cm] at (2.1,\h+1.15) {\textbf{Solución}\\Recomendaciones personalizadas, rutas de aprendizaje dinámicas, feedback automático y tutor virtual.};

    \draw (\w,\h) rectangle (\w*2,\h*2);
    \node[text width=3.8cm] at (\w*1.5,\h+1.15) {\textbf{Canales}\\Web, app móvil, alianzas con universidades y empresas.};

    \draw (\w*2,\h) rectangle (\w*3,\h*2);
    \node[text width=3.8cm] at (\w*2.5,\h+1.15) {\textbf{Métricas Clave}\\Tasa de finalización de cursos, usuarios activos, satisfacción del usuario, retención mensual.};

    % Fila inferior (3 bloques)
    \draw (0,\h*2) rectangle (\w,\h*3);
    \node[text width=3.8cm] at (2.1,\h*2+1.15) {\textbf{Estructura de Costos}\\Desarrollo y mantenimiento de la plataforma, licencias de IA, marketing, soporte técnico.};

    \draw (\w,\h*2) rectangle (\w*2,\h*3);
    \node[text width=3.8cm] at (\w*1.5,\h*2+1.15) {\textbf{Fuentes de Ingresos}\\Suscripciones mensuales, licencias B2B, cursos premium, anuncios.};

    \draw (\w*2,\h*2) rectangle (\w*3,\h*3);
    \node[text width=3.8cm] at (\w*2.5,\h*2+1.15) {\textbf{Ventaja Injusta}\\Algoritmo propio de personalización, base de datos de aprendizaje, alianzas estratégicas con instituciones.};
\end{tikzpicture}
\end{center}

\subsection{Validación}

Para validar el problema y la solución, se pueden realizar entrevistas y encuestas a estudiantes universitarios y profesionales para conocer sus necesidades y percepciones sobre la personalización del aprendizaje. También es posible desarrollar un prototipo mínimo de la plataforma y probarlo con un grupo reducido de usuarios, recolectando su retroalimentación sobre la utilidad y la experiencia de personalización mediante IA. Adicionalmente, se pueden analizar métricas como la tasa de interés, la intención de uso y la satisfacción de los usuarios durante las pruebas iniciales.
\section*{Conclusiones}
El uso de herramientas como el Business Model Canvas y el Lean Canvas permite a los equipos estructurar
y visualizar sus ideas de negocio de manera efectiva. Estos modelos no solo facilitan la identificación de los elementos clave del negocio, sino que también ayudan a validar hipótesis y ajustar estrategias basadas en la retroalimentación del mercado. La innovación en modelos de negocio es esencial para adaptarse a un entorno empresarial en constante cambio y para satisfacer las necesidades emergentes de los clientes.

\end{document}
