\documentclass{article}
\usepackage[a4paper, top=3cm, bottom=2.5cm, left=2.5cm, right=2.5cm]{geometry} % Ajuste de márgenes
\usepackage[spanish]{babel}
\usepackage[utf8]{inputenc}
\usepackage{tabularx}
\usepackage{tikz}
\usepackage{titling}
\usepackage{graphicx}
\usepackage{fancyhdr}
\usepackage{amsmath}
\usepackage{amssymb}
\usepackage{multicol}
\usepackage{cancel}
\usepackage{pgfplots}
\usepackage{hyperref}
\usepackage{bookmark}
\pgfplotsset{compat=1.18}
\usepackage{titlesec} % Para personalizar títulos
\usepackage{tocloft}  % Para mejorar el índice
\usepackage{setspace} % Para controlar el espaciado

\usepackage{xcolor}
\usepackage{enumitem}

\definecolor{headerblue}{RGB}{50,90,140}
\definecolor{responsegray}{RGB}{80,80,80}



% Configuración de Fancyhdr para encabezados y pies de página
\pagestyle{fancy}
\fancyhf{}
\fancyhead[L]{\includegraphics[width=2cm]{assets/logo-utp.png}}
\fancyhead[R]{\textbf{Diseño de Productos y Servicios}}

\fancyfoot[R]{\thepage} % Número de página alineado a la derecha

% Ajustes de espaciado entre párrafos y márgenes superiores
\setlength{\parskip}{1.5em}
\setlength{\parindent}{0pt}
\setlength{\headheight}{17.26935pt} % Altura del encabezado
\addtolength{\topmargin}{-2.26935pt} % Compensar el aumento de la altura del encabezado
\setlength{\textheight}{23cm}  % Ajusta el alto del texto

% Definición de comandos personalizados
\newcommand{\SubItem}[1]{
    {\setlength\itemindent{15pt} \item[-] #1}
}

% Título del documento con mejor control de espaciado
\title{
  \includegraphics[width=5cm]{./assets/logo-utp.png} \\
  \vspace{1cm}
  \textbf{Universidad Tecnológica del Perú} \\
  \vspace{2cm}
  \textbf{Informe de mejora contínua} \\
  \vspace{1cm}
  \large \textbf{Para el curso de Diseño de Productos y Servicios.}
}
\author{
  \textbf{Luis Huatay Salcedo.} \\
  \textbf{Garcia Chumpitaz Cindel Roxell.} \\
  \textbf{Díaz Benítez, Fernando Raúl.} \\
  \textbf{Quispe Fernandez, Bryan Alexander.}
}


\begin{document}
\maketitle
\begin{center}
  Sección 44698
\end{center}
\thispagestyle{empty}
\begin{center}
  Mg. Marcos Teodoro Yerren Huima  
\end{center}
\restoregeometry

% \newpage

% \begin{center}
%   \textbf{\Large Índice}
% \end{center}
% \vspace{0.5cm} % Espacio entre título y contenido

% \begin{spacing}{1.15} % Espaciado personalizado para mayor legibilidad
%   \noindent
%   \begin{enumerate}
%     \item Introducción
%   %   \item Problemática
%   %   \item Objetivo general
%   %   \begin{enumerate}
%   %     \item Objetivos específicos
%   %   \end{enumerate}
%   %   \item Términos estadísticos
%   %   \item Recolección de información
%     \end{enumerate}
% \end{spacing}

\newpage


%-------------------------------------------------
% Perfil del usuario
%-------------------------------------------------
\section{Sobre la mejora contínua de la plataforma UTP}

Los cambios realizados en la plataforma UTP tienen como objetivo mejorar la experiencia del usuario, facilitando el acceso a la información académica y administrativa. A continuación, se presenta un análisis detallado de los aspectos positivos, las áreas de mejora y las acciones concretas que se han implementado.

\subsection*{Aspectos de soluciones que pueden seguir mejorando:}

Con respecto a las soluciones propuestas, estas pretenden ser un punto de partida para una mejora continua de la plataforma. Aunque se han implementado cambios significativos, es importante seguir evaluando y ajustando las funcionalidades para adaptarse a las necesidades cambiantes de los usuarios. La retroalimentación constante de los estudiantes y el personal académico es crucial para identificar nuevas áreas de mejora y garantizar que la plataforma siga siendo una herramienta efectiva y eficiente.

\subsection*{Dos ideas concretas de mejora:}

Para mejorar la plataforma UTP, se proponen dos ideas concretas:
\begin{itemize}
  \item \textbf{Implementación de un sistema de navegación intuitivo:} Rediseñar la estructura de navegación para que los usuarios puedan encontrar fácilmente la información que necesitan. Esto podría incluir una barra de navegación más clara y accesible, así como una organización más lógica de las secciones.
  \item \textbf{Optimización del rendimiento:} Mejorar la velocidad de carga de las páginas y la estabilidad técnica de la plataforma. Esto podría implicar la optimización de bases de datos, servidores y el uso de tecnologías más eficientes para garantizar una experiencia fluida y sin interrupciones.
\end{itemize}

\subsection*{¿Por qué se considera que estas mejoras beneficiarán a los usuarios?}
Estas mejoras se consideran beneficiosas para los usuarios porque:
\begin{itemize}
  \item \textbf{Facilitan el acceso a la información:} Un sistema de navegación intuitivo permitirá a los usuarios encontrar rápidamente lo que necesitan, reduciendo la frustración y el tiempo perdido en la búsqueda de información.
  \item \textbf{Mejoran la experiencia general:} La optimización del rendimiento hará que la plataforma sea más ágil y receptiva, lo que contribuirá a una experiencia más satisfactoria y eficiente para todos los usuarios.
  \item \textbf{Aumentan la satisfacción del usuario:} Al abordar las preocupaciones actuales sobre la usabilidad y el rendimiento, se espera que los usuarios se sientan más satisfechos con la plataforma, lo que puede llevar a un mayor uso y aceptación de las herramientas digitales ofrecidas por la UTP.
  \item \textbf{Fomentan la adopción de la plataforma:} Al mejorar la usabilidad y el rendimiento, se espera que más estudiantes y personal académico utilicen la plataforma de manera regular, lo que puede llevar a una mayor eficiencia en la gestión académica y administrativa.
\end{itemize}
\end{document}