\documentclass{article}
\usepackage[a4paper, top=3cm, bottom=2.5cm, left=2.5cm, right=2.5cm]{geometry} % Ajuste de márgenes
\usepackage[spanish]{babel}
\usepackage[utf8]{inputenc}
\usepackage{tikz}
\usepackage{titling}
\usepackage{graphicx}
\usepackage{fancyhdr}
\usepackage{amsmath}
\usepackage{amssymb}
\usepackage{multicol}
\usepackage{cancel}
\usepackage{pgfplots}
\usepackage{hyperref}
\usepackage{bookmark}
\pgfplotsset{compat=1.18}
\usepackage{titlesec} % Para personalizar títulos
\usepackage{tocloft}  % Para mejorar el índice
\usepackage{setspace} % Para controlar el espaciado

\usepackage{xcolor}
\usepackage{enumitem}

\definecolor{headerblue}{RGB}{50,90,140}
\definecolor{responsegray}{RGB}{80,80,80}



% Configuración de Fancyhdr para encabezados y pies de página
\pagestyle{fancy}
\fancyhf{}
\fancyhead[L]{\includegraphics[width=2cm]{assets/logo-utp.png}}
\fancyhead[R]{\textbf{Diseño de Productos y Servicios}}

\fancyfoot[R]{\thepage} % Número de página alineado a la derecha

% Ajustes de espaciado entre párrafos y márgenes superiores
\setlength{\parskip}{1.5em}
\setlength{\parindent}{0pt}
\setlength{\headheight}{17.26935pt} % Altura del encabezado
\addtolength{\topmargin}{-2.26935pt} % Compensar el aumento de la altura del encabezado
\setlength{\textheight}{23cm}  % Ajusta el alto del texto

% Definición de comandos personalizados
\newcommand{\SubItem}[1]{
    {\setlength\itemindent{15pt} \item[-] #1}
}

% Título del documento con mejor control de espaciado
\title{
  \includegraphics[width=5cm]{./assets/logo-utp.png} \\
  \vspace{1cm}
  \textbf{Universidad Tecnológica del Perú} \\
  \vspace{2cm}
  \textbf{Entrevistando a un usuarios extremo} \\
  \vspace{1cm}
  \large \textbf{Para el curso de Diseño de Productos y Servicios.}
}
\author{
  \textbf{Luis Huatay Salcedo.} \\
  \textbf{Garcia Chumpitaz Cindel Roxell.} \\
  \textbf{Díaz Benítez, Fernando Raúl.} \\
  \textbf{Quispe Fernandez, Bryan Alexander.}
}


\begin{document}
\maketitle
\begin{center}
  Sección 44698
\end{center}
\thispagestyle{empty}
\begin{center}
  Mg. Marcos Teodoro Yerren Huima  
\end{center}
\restoregeometry

% \newpage

% \begin{center}
%   \textbf{\Large Índice}
% \end{center}
% \vspace{0.5cm} % Espacio entre título y contenido

% \begin{spacing}{1.15} % Espaciado personalizado para mayor legibilidad
%   \noindent
%   \begin{enumerate}
%     \item Introducción
%   %   \item Problemática
%   %   \item Objetivo general
%   %   \begin{enumerate}
%   %     \item Objetivos específicos
%   %   \end{enumerate}
%   %   \item Términos estadísticos
%   %   \item Recolección de información
%     \end{enumerate}
% \end{spacing}

\newpage


%-------------------------------------------------
% Perfil del usuario
%-------------------------------------------------
\section*{Perfil del usuario entrevistado}
\begin{center}
  \begin{tabular}{p{0.3\linewidth} p{0.65\linewidth}}
    \textbf{Nombre y edad:}   & Sofía Ángeles V. Rasputín, 68 años \\
    \textbf{Condición:}       & Persona mayor, visión reducida, baja alfabetización digital \\
    \textbf{Producto:}        & WhatsApp (mensajería) \\
  \end{tabular}
\end{center}

% ----------------------------------------
% ENTREVISTA INFORMAL – USUARIO EXTREMO
% ----------------------------------------
\section*{Entrevista informal – Usuario extremo}

\subsection*{Preguntas y respuestas}
\begin{enumerate}
  \item \textbf{¿Qué producto o aplicación digital ha usado recientemente?}\\
    He usado WhatsApp para comunicarme con mis hijos y nietos.
  
  \item \textbf{¿Qué dificultades encontró al usarlo?}\\
    Las letras son muy pequeñas. A veces me mandan audios y no sé cómo responder. También me confundo cuando hay muchos mensajes en un grupo familiar.
  
  \item \textbf{¿Qué esperaba lograr con esa aplicación?}\\
    Solo quiero hablar con mi familia, mandar mensajes sencillos y fotos, pero me cuesta saber dónde está cada cosa ya que no estoy acostumbrada.
  
  \item \textbf{¿Qué le gustaría que fuera diferente para que le sea más fácil de usar?}\\
    Me gustaría que tuviera letras grandes, explicaciones más claras, y que no tenga tantas funciones juntas. A veces solo quiero enviar un mensaje, pero toco otra cosa sin querer.
\end{enumerate}

% ----------------------------------------
% DESCRIPCIÓN DE HALLAZGOS
% ----------------------------------------
\section*{Descripción de hallazgos}

\begin{itemize}
  \item \textbf{¿Qué dificultades encontró?}\\
        Sofía experimenta problemas de legibilidad debido al tamaño reducido de las letras, se siente desorientada en grupos con muchos mensajes y no sabe cómo reproducir ni responder a los audios que recibe.
        
  \item \textbf{¿Qué necesidades o expectativas tiene?}\\
        Espera una interfaz más clara y simple que le permita enviar mensajes y fotos sin confusiones, así como una forma intuitiva de gestionar y contestar audios. Necesita textos de mayor tamaño y explicaciones paso a paso de las funciones básicas.
        
  \item \textbf{¿Qué mejora propondrías para facilitarle la experiencia?}\\
        Un “Modo Simplificado” con tipografía XXL y alto contraste, solo tres botones principales (mensaje, audio y foto) etiquetados con texto descriptivo, y un asistente inicial que la guíe de forma interactiva por cada acción.
\end{itemize}

%-------------------------------------------------
% Resumen de lo aprendido
%-------------------------------------------------
\section*{Resumen de lo aprendido}
\begin{itemize}
  \item \textbf{Texto pequeño:} Le cuesta leer mensajes con el tamaño de letra actual.
  \item \textbf{Interfaz sobrecargada:} Confusión al haber muchas funciones en pantalla.
  \item \textbf{Audios:} No sabe cómo reproducir ni responder grabaciones.
  \item \textbf{Falta de guía:} Necesita explicaciones claras y un tutorial paso a paso.
  \item \textbf{Errores involuntarios:} Presiona opciones erróneas al intentar enviar mensajes simples.
\end{itemize}

%-------------------------------------------------
% Idea de mejora
%-------------------------------------------------
\section*{Idea inicial de mejora}
\begin{itemize}
  \item \textbf{Modo Simplificado:} Interfaz reducida a tres acciones: enviar mensaje, enviar audio y enviar foto.
  \item \textbf{Tipografía XXL:} Letras de gran tamaño con alto contraste.
  \item \textbf{Iconos con etiqueta:} Cada botón con ícono y texto descriptivo (“Enviar mensaje”).
  \item \textbf{Tutorial interactivo:} Asistente paso a paso la primera vez que abra la app.
  \item \textbf{Opción de retorno:} Botón para volver al modo completo cuando lo desee.
\end{itemize}

\end{document}