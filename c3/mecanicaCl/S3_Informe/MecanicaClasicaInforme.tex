\documentclass{article}
\usepackage[a4paper, top=3cm, bottom=2.5cm, left=2.5cm, right=2.5cm]{geometry} % Ajuste de márgenes
\usepackage[spanish]{babel}
\usepackage[utf8]{inputenc}
\usepackage{tikz}
\usepackage{titling}
\usepackage{graphicx}
\usepackage{fancyhdr}
\usepackage{amsmath}
\usepackage{amssymb}
\usepackage{multicol}
\usepackage{cancel}
\usepackage{pgfplots}
\usepackage{hyperref}
\pgfplotsset{compat=1.18}
\usepackage{titlesec} % Para personalizar títulos
\usepackage{tocloft}  % Para mejorar el índice
\usepackage{setspace} % Para controlar el espaciado

% Configuración de Fancyhdr para encabezados y pies de página
\pagestyle{fancy}
\fancyhf{}
\fancyhead[L]{\includegraphics[width=2cm]{assets/logo-utp.png}}
\fancyhead[R]{\textbf{Mecánica Clásica}}

\fancyfoot[R]{\thepage} % Número de página alineado a la derecha

% Ajustes de espaciado entre párrafos y márgenes superiores
\setlength{\parskip}{1.5em}
\setlength{\parindent}{0pt}
\setlength{\headheight}{17.26935pt} % Altura del encabezado
\addtolength{\topmargin}{-2.26935pt} % Compensar el aumento de la altura del encabezado
\setlength{\textheight}{23cm}  % Ajusta el alto del texto

% Definición de comandos personalizados
\newcommand{\SubItem}[1]{
    {\setlength\itemindent{15pt} \item[-] #1}
}

% Título del documento con mejor control de espaciado
\title{
  \includegraphics[width=5cm]{./assets/logo-utp.png} \\
  \vspace{1cm}
  \textbf{Universidad Tecnológica del Perú} \\
  \vspace{2cm}
  \textbf{RANDOMBEAST: Generación de números aleatorios basada en entropía eólica usando un Strandbeest} \\
  \vspace{1cm}
  \large \textbf{Para el curso de Mecánica Clásica.}
}
\author{
  \textbf{Luis Huatay Salcedo.} \\
  \texttt{hsluis4326@gmail.com} \\
  \texttt{U24218809 - 24229} \\
  \textbf{Cristian Miguel Cristóbal Blas.} \\
}


\begin{document}
\maketitle
\thispagestyle{empty}
\begin{center}
  Mg. Jonathan Joas Zapata Campos.  
\end{center}
\restoregeometry

\newpage

\begin{center}
  \textbf{\Large Índice}
\end{center}
\vspace{0.5cm} % Espacio entre título y contenido

\begin{spacing}{1.15} % Espaciado personalizado para mayor legibilidad
  \noindent
  \begin{enumerate}
    \item Introducción
  %   \item Problemática
  %   \item Objetivo general
  %   \begin{enumerate}
  %     \item Objetivos específicos
  %   \end{enumerate}
  %   \item Términos estadísticos
  %   \item Recolección de información
    \end{enumerate}
\end{spacing}

\newpage
\vspace*{\fill}
\section{Introducción}
La aleatoriedad es un concepto fundamental en la teoría de la probabilidad y la estadística, en la informática, ciencias de la computación, criptografía, etc. Hoy en día nuestras cuentas bancarias, códigos de seguridad y datos personales dependen de la aleatoriedad para permanecer seguros. Para esto existen muchos algoritmos matemáticos que pretenden dar una solución a esto en la generación de números aleatorios. Sin embargo es matemáticamente imposible para una computadora crear un número verdaderamente aleatorio.

Sin emabrgo, existen medios físicos que pueden ser utilizados para generar aleatoriedad, en lo que los expertos llaman \textit{entropía}. La entropía es una medida de la incertidumbre o el desorden en un sistema. En el contexto de la generación de números aleatorios, la entropía se refiere a la cantidad de información impredecible que se puede extraer de un sistema físico. Cuanto más impredecible sea el sistema, mayor será su entropía.

Es aquí donde entra en juego el Strandbeest \cite{strandbeest2024}, una obra maestra de la ingeniería y el arte creada por el artista e ingeniero Theo Jansen Strandbeest. El Strandbeest es una criatura mecánica que camina utilizando energía eólica, y su diseño se basa en principios de la mecánica clásica. La idea detrás de este proyecto es utilizar el movimiento del Strandbeest para generar números aleatorios a partir de la entropía eólica.

\vspace*{\fill}

\newpage



\newpage

\bibliographystyle{plain} % Estilo (APA, IEEE, etc.)
\bibliography{referencias} % Nombre del archivo .bib (sin extensión)  \printbibliography

\end{document}