\vspace*{\fill}

\section{Introducción}

  Los avances tecnológicos en el ámbito de la visión por computadora han revolucionado la forma en que las máquinas interpretan y entienden el mundo visual que las rodea. Entre las diversas tareas que conforman este campo, la detección de objetos se destaca como una de las más fundamentales y desafiantes. Esta tarea implica identificar y localizar instancias de objetos específicos dentro de imágenes o secuencias de video, lo que tiene aplicaciones prácticas en áreas tan diversas como la vigilancia, la conducción autónoma, la robótica, y la realidad aumentada. 

  Según \cite{Zou2019}, la detección de objetos ha experimentado una evolución significativa a lo largo de las últimas décadas, impulsada en gran medida por los avances en el aprendizaje profundo y las redes neuronales convolucionales. Desde los primeros enfoques basados en características manuales hasta los métodos modernos que emplean arquitecturas profundas, la capacidad de las máquinas para reconocer y localizar objetos ha mejorado drásticamente.

  En ese sentido, este proyecto se centra en la implementación y evaluación de un sistema de detección y clasificación de vegetales. Este sistema utilizará componentes de hardware accesibles como controladores y sensores de bajo costo. Además, se hace especial énfasis en los fundamentos teóricos del electromagnetismo, que son esenciales para comprender el funcionamiento de los sensores utilizados en la captura de imágenes.

\vspace*{\fill}