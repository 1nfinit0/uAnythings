\section{Descripción de la empresa y Área afectada por el proyecto a implementar}

\subsection{Nombre de la empresa y/o Área}
\textbf{Empresa:} Global GIS Service (GGS) \\
\textbf{Área afectada:} Investigación y Desarrollo (I+D)

\subsection{Misión y Visión de la empresa}

\textbf{Misión:} \\
Desarrollar soluciones geoespaciales innovadoras que optimicen la toma de decisiones estratégicas en organizaciones públicas y privadas, integrando tecnología GIS con metodologías ágiles y sostenibles.

\textbf{Visión:} \\
Ser líderes en Latinoamérica en el desarrollo de plataformas GIS interoperables, escalables y centradas en el usuario, contribuyendo a la transformación digital del sector educativo, gubernamental y empresarial.

\subsection{Objetivos Estratégicos de la empresa y/o Área}

\textbf{Objetivos Estratégicos de Global GIS Service (GGS):}
\begin{itemize}
    \item Expandir la adopción de soluciones GIS en sectores clave como educación, gobierno y servicios públicos.
    \item Garantizar la interoperabilidad entre plataformas GIS y sistemas de terceros.
    \item Fortalecer la relación con clientes mediante procesos de validación colaborativos y comunicación efectiva.
    \item Promover la capacitación continua de usuarios finales para asegurar el uso eficiente de las soluciones implementadas.
\end{itemize}

\textbf{Objetivos Estratégicos del Área de I+D:}
\begin{itemize}
    \item Diseñar una metodología de trabajo sostenible y replicable para el desarrollo de proyectos GIS.
    \item Fomentar la innovación mediante la incorporación de tecnologías emergentes en los procesos de desarrollo.
    \item Asegurar la calidad técnica de los entregables a través de revisiones periódicas y pruebas de integración.
    \item Generar conocimiento interno que permita escalar soluciones de manera eficiente en nuevos mercados.
\end{itemize}

\subsection{Alineamiento del proyecto con los objetivos de la empresa}

El proyecto \textbf{Global GIS Service GGS} se alinea directamente con los objetivos estratégicos de la empresa y del área de I+D en los siguientes aspectos:

\begin{itemize}
    \item \textbf{Interoperabilidad tecnológica:} El proyecto aborda la integración entre plataformas GIS y sistemas del cliente, lo cual responde al objetivo de garantizar compatibilidad y escalabilidad.
    \item \textbf{Metodología sostenible:} La implementación de sesiones prácticas, manuales claros y validaciones técnicas contribuye a establecer una metodología de trabajo replicable en futuros proyectos.
    \item \textbf{Gestión del conocimiento:} Las reuniones técnicas mensuales y los informes especializados permiten documentar aprendizajes y buenas prácticas, fortaleciendo el capital intelectual del área de I+D.
    \item \textbf{Transformación digital:} Al automatizar procesos de validación, capacitación y comunicación, el proyecto impulsa la digitalización de la gestión geoespacial en el cliente, alineándose con la visión de liderazgo regional en soluciones GIS.
\end{itemize}
