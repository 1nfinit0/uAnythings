\section{Proceso de Iniciación}

En el proceso de inicación, se definen los objetivos del proyecto, su alcance inicial, la gestión de interesados y se elabora el acta de constitución del proyecto. Esto establece las bases para la planificación y ejecución del mismo.

\subsection{Objetivos del Proyecto}
\begin{itemize}
    \item Implementar una solución GIS interoperable que permita integrar plataformas geoespaciales con los sistemas del cliente.
    \item Establecer una metodología de trabajo sostenible y replicable para futuros desarrollos GIS desde el área de I+D.
    \item Mejorar la calidad de los datos y la eficiencia operativa mediante validaciones técnicas y automatización de procesos.
    \item Fortalecer la relación con el cliente a través de una comunicación efectiva y validación continua de entregables.
\end{itemize}

\subsection{Alcance Inicial del Proyecto}

El proyecto \textbf{Global GIS Service GGS} tiene como alcance inicial el diseño e implementación de servicios de consultoría en Sistemas de Información Geográfica (GIS) para clientes en sectores público y privado. El alcance incluye:
\begin{itemize}
    \item Análisis de necesidades y requisitos del cliente.
    \item Desarrollo de soluciones GIS personalizadas.
    \item Integración de plataformas GIS con sistemas existentes del cliente.
    \item Capacitación básica a usuarios finales.
    \item Documentación técnica y manuales de usuario.
\end{itemize}

\subsubsection{Registro de Interesados}

\begin{table}[H]
\centering
\renewcommand{\arraystretch}{1.3}
\begin{tabular}{|p{4cm}|p{10cm}|}
\hline
\textbf{Atributo} & \textbf{Valores} \\
\hline
Rol en Proyecto & Patrocinador, Especialista Técnico \\
\hline
Nombre & Dra. Rosmery R., Ing. Marco T. \\
\hline
Cargo & Directora de Investigación y Proyectos Estratégicos, Especialista GIS \\
\hline
Área & Dirección, Área Técnica \\
\hline
Principal Requerimiento & Alineación estratégica del proyecto con los objetivos institucionales, Interoperabilidad entre plataformas GIS y sistemas del cliente \\
\hline
Influencia (Poder / Interés) & Alta / Alta, Media / Alta \\
\hline
Estrategia & Reuniones periódicas, reportes ejecutivos, validación de entregables; Involucramiento en revisiones técnicas, pruebas de integración \\
\hline
¿Aprueba Alcance / Plan? & Sí, No \\
\hline
Anexo & A1, A2 \\
\hline
Contacto & 999-123-456, 999-987-654 \\
\hline
\end{tabular}
\caption{Registro de Interesados del Proyecto Global GIS Service GGS (orientación horizontal)}
\end{table}

\subsubsection{Acta de Constitución del Proyecto}

\begin{table}[H]
\centering
\renewcommand{\arraystretch}{1.4}
\begin{tabular}{|p{5cm}|p{10cm}|}
\hline
\textbf{Componente} & \textbf{Descripción} \\
\hline
Equipo & Global GIS Service GGS \\
\hline
Nombre del Proyecto & Consultoría Integral en Soluciones GIS \\
\hline
Gerente del Proyecto & Luis Huatay – Nivel de autoridad: Alto. Reporta a la Dirección de Proyectos Estratégicos. Pertenece a Global GIS Service. \\
\hline
Patrocinador del Proyecto & Dra. Rosmery R. – Directora de Investigación y Proyectos Estratégicos \\
\hline
Descripción del Proyecto & Diseño e implementación de servicios de consultoría especializados en Sistemas de Información Geográfica (GIS), orientados a mejorar la toma de decisiones mediante análisis espacial, interoperabilidad tecnológica y capacitación de usuarios. \\
\hline
Justificación del Proyecto & Alta demanda en sectores público y privado por soluciones geoespaciales integradas. Necesidad de posicionar a la empresa como referente regional en consultoría GIS. Oportunidad de generar ventajas competitivas mediante visualización de datos y transformación digital. \\
\hline
Objetivos del Proyecto y Criterios de Éxito & 
\begin{itemize}
    \item Brindar servicios GIS personalizados que satisfagan necesidades específicas (criterio: contratos ejecutados con éxito).
    \item Entregar soluciones en un plazo máximo de 8 semanas por cliente (criterio: cumplimiento del cronograma).
    \item Superar el 90\% de satisfacción del cliente en encuestas de cierre.
\end{itemize} \\
\hline
Requerimientos Principales (Alto nivel) & Plataformas GIS, personal especializado en análisis espacial y programación, documentación técnica, capacitación básica a usuarios finales. \\
\hline
Riesgos Principales (Alto nivel) & Interoperabilidad limitada entre sistemas GIS y plataformas del cliente, dependencia de datos externos, restricciones presupuestarias. \\
\hline
Resumen del Cronograma de Hitos & 
\begin{itemize}
    \item Septiembre - Octubre: Definición y constitución de la empresa.
    \item Noviembre - Diciembre: Desarrollo de la oferta de servicios y marketing inicial.
\end{itemize} \\
\hline
Presupuesto Resumido (Orden de Magnitud) & S/. 20,000 en 4 meses \\
\hline
Supuestos & Disponibilidad de datos geoespaciales confiables, colaboración activa de los clientes, acceso a infraestructura tecnológica mínima. \\
\hline
Restricciones & Presupuesto limitado, plazos ajustados, disponibilidad de personal especializado. \\
\hline
Interesados & Ingeniero de Software especializado en GIS, Directora de Investigación y Proyectos Estratégicos, PMO, usuarios finales, equipos técnicos del cliente. \\
\hline
Requerimientos de Aprobación del Proyecto & Aprobación formal del patrocinador y del equipo directivo. Validación del acta de constitución por unanimidad. \\
\hline
\end{tabular}
\caption{Acta de Constitución del Proyecto: Global GIS Service GGS}
\end{table}


