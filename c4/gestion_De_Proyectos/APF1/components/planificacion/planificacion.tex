\section{Planificación}

  En esta sección se detallan los procesos de planificación del proyecto, incluyendo la gestión de la integración, alcance, cronograma, costes, calidad, recursos, comunicaciones, riesgos y adquisiciones. Cada uno de estos aspectos es fundamental para asegurar que el proyecto se ejecute de manera eficiente y efectiva, alineándose con los objetivos estratégicos de la empresa.

    Se utilizarán herramientas y técnicas del PMBOK para desarrollar un plan de proyecto integral que guíe todas las fases posteriores del mismo. Esto incluye la definición clara del alcance del proyecto, la elaboración de un cronograma detallado con hitos clave, la estimación y control de costes, así como la identificación y mitigación de riesgos potenciales.

    Además, se establecerán mecanismos de comunicación efectivos para mantener informados a todos los interesados y asegurar una gestión adecuada de los recursos humanos y materiales necesarios para la ejecución del proyecto.

    \subsection{Gestión de la Integración del Proyecto}
    
    \textbf{Objetivo:}
      Coordinar de manera efectiva todos los componentes del proyecto, asegurando la coherencia entre los procesos de iniciación, planificación, ejecución, monitoreo, control y cierre, para cumplir con los objetivos estratégicos de la empresa.

      \textbf{Procesos Clave}
      \begin{itemize}
          \item \textbf{Desarrollar el Acta de Constitución del Proyecto:} Formaliza el inicio del proyecto y define los objetivos, alcance y responsables.
          \item \textbf{Desarrollar el Plan para la Dirección del Proyecto:} Integra los planes subsidiarios (alcance, cronograma, costos, calidad, riesgos, comunicaciones, interesados).
          \item \textbf{Dirigir y Gestionar el Trabajo del Proyecto:} Ejecuta las actividades planificadas, gestiona recursos, genera entregables y asegura la calidad.
          \item \textbf{Gestionar el Conocimiento del Proyecto:} Captura, comparte y reutiliza conocimiento tácito y explícito para mejorar resultados y fomentar el aprendizaje organizacional.
          \item \textbf{Monitorear y Controlar el Trabajo del Proyecto:} Supervisa el desempeño, identifica desviaciones y toma acciones correctivas.
          \item \textbf{Realizar el Control Integrado de Cambios:} Evalúa y aprueba cambios que afecten el alcance, tiempo, costo o calidad.
          \item \textbf{Cerrar el Proyecto o Fase:} Verifica la aceptación de entregables, documenta lecciones aprendidas y formaliza el cierre.
      \end{itemize}

      \textbf{Entradas Principales}
      \begin{itemize}
          \item Acta de Constitución del Proyecto
          \item Plan para la Dirección del Proyecto
          \item Documentos del Proyecto
          \item Factores Ambientales de la Empresa
          \item Activos de los Procesos de la Organización
      \end{itemize}

      \textbf{Herramientas y Técnicas}
      \begin{itemize}
          \item Juicio de Expertos
          \item Sistemas de Información para la Gestión del Proyecto
          \item Reuniones
          \item Gestión del Conocimiento e Información
          \item Habilidades Interpersonales: escucha activa, facilitación, liderazgo, creación de relaciones, conciencia política
      \end{itemize}

      \textbf{Salidas}
      \begin{itemize}
          \item Entregables
          \item Datos de Desempeño del Trabajo
          \item Solicitudes de Cambio
          \item Registro de Incidentes
          \item Actualizaciones al Plan para la Dirección del Proyecto
          \item Actualizaciones a los Documentos del Proyecto
          \item Actualizaciones a los Activos de los Procesos de la Organización
          \item Registro de Lecciones Aprendidas
      \end{itemize}

    \subsection{Gestión del Alcance del Proyecto}

    Se desarrolla la EDT, un EDT es una descomposición jerárquica del trabajo del proyecto que define y organiza el alcance total del proyecto. La EDT descompone el proyecto en componentes más pequeños y manejables, facilitando la planificación, ejecución y control del proyecto.

    \begin{figure}[H]
      \centering
      \includegraphics[width=1.5\textwidth,angle=90,trim=1cm 1cm 1cm 1cm,clip]{assets/EDT.png}
      \caption{Ejemplo de Estructura de Desglose del Trabajo (EDT) del proyecto.}
      \label{fig:edt}
    \end{figure}

    \subsubsection{Gestión del Cronograma del Proyecto}

    \textbf{Objetivo}
    Planificar, desarrollar, monitorear y controlar el cronograma del proyecto para asegurar el cumplimiento de los plazos establecidos, optimizando recursos y gestionando desviaciones de manera efectiva.

    \begin{multicols}{2}
      \textbf{Procesos Clave}
    \begin{itemize}
        \item \textbf{Planificar la Gestión del Cronograma:} Definir cómo se desarrollará, gestionará y controlará el cronograma.
        \item \textbf{Definir las Actividades:} Identificar las tareas específicas necesarias para producir los entregables del proyecto.
        \item \textbf{Secuenciar las Actividades:} Establecer el orden lógico de ejecución de las actividades.
        \item \textbf{Estimar la Duración de las Actividades:} Determinar el tiempo requerido para completar cada actividad.
        \item \textbf{Desarrollar el Cronograma:} Integrar actividades, duraciones, dependencias y recursos en un cronograma realista.
        \item \textbf{Controlar el Cronograma:} Monitorear el progreso, actualizar el cronograma y gestionar cambios a la línea base.
    \end{itemize}

    \textbf{Entradas}
    \begin{itemize}
        \item Plan para la Dirección del Proyecto
        \item Línea base del cronograma
        \item Documentos del proyecto (calendarios, estimaciones, riesgos)
        \item Datos de desempeño del trabajo
        \item Activos de los procesos de la organización
    \end{itemize}

    \textbf{Herramientas y Técnicas}
    \begin{itemize}
        \item Análisis de datos: valor ganado (SV, SPI), tendencias, variaciones, escenarios
        \item Método de la ruta crítica
        \item Sistema de información para la dirección de proyectos
        \item Optimización de recursos
        \item Adelantos y retrasos
        \item Compresión del cronograma
        \item Gráfica de trabajo pendiente por iteración
    \end{itemize}

    \textbf{Salidas}
    \begin{itemize}
        \item Informes de desempeño del trabajo
        \item Pronósticos del cronograma
        \item Solicitudes de cambio
        \item Actualizaciones al plan para la dirección del proyecto (línea base, costos, desempeño)
        \item Actualizaciones a los documentos del proyecto (cronograma, calendario de recursos, riesgos)
    \end{itemize}

    \textbf{Indicadores de Control}
    \begin{itemize}
        \item \textbf{SV (Schedule Variance):} Variación entre el trabajo planificado y el realizado.
        \item \textbf{SPI (Schedule Performance Index):} Índice de eficiencia temporal del proyecto.
        \item \textbf{Duración restante por actividad:} Estimación actualizada del tiempo requerido.
        \item \textbf{Gráfica de iteración:} Seguimiento visual del trabajo pendiente vs. ideal.
    \end{itemize}

    \end{multicols}

    \subsubsection{Gestión de los Costes del Proyecto}

      \begin{multicols}{2}
        \textbf{Objetivo}
        Planificar, estimar, controlar y actualizar los costos del proyecto para asegurar que se mantenga dentro del presupuesto aprobado, garantizando eficiencia financiera y toma de decisiones informada.

        \textbf{Procesos Clave}
        \begin{itemize}
            \item \textbf{Planificar la Gestión de Costes:} Definir cómo se estimarán, gestionarán y controlarán los costos del proyecto.
            \item \textbf{Estimar los Costos:} Determinar el costo aproximado de los recursos necesarios para completar cada actividad.
            \item \textbf{Determinar el Presupuesto:} Agregar los costos estimados para establecer una línea base de costos.
            \item \textbf{Controlar los Costos:} Monitorear el desempeño del proyecto frente a la línea base, gestionar desviaciones y actualizar proyecciones.
        \end{itemize}

        \textbf{Entradas}
        \begin{itemize}
            \item Plan para la Dirección del Proyecto
            \item Línea base de costos
            \item Requisitos de financiamiento del proyecto
            \item Documentos del proyecto (estimaciones, riesgos, supuestos)
            \item Datos de desempeño del trabajo
            \item Activos de los procesos de la organización
        \end{itemize}

        \textbf{Herramientas y Técnicas}
        \begin{itemize}
            \item Juicio de expertos
            \item Análisis de datos:
            \begin{itemize}
                \item Valor ganado (EV)
                \item Variación de costos (CV = EV - AC)
                \item Variación del cronograma (SV = EV - PV)
                \item Índice de desempeño del costo (CPI = EV / AC)
                \item Índice de desempeño del cronograma (SPI = EV / PV)
                \item Estimación a la conclusión (EAC)
                \item Estimación hasta la conclusión (ETC)
                \item Variación a la conclusión (VAC = BAC - EAC)
                \item Índice de desempeño del trabajo por completar (TCPI)
            \end{itemize}
            \item Sistema de información para la dirección de proyectos
            \item Análisis de tendencias y reservas
        \end{itemize}

        \textbf{Salidas}
        \begin{itemize}
            \item Informes de desempeño del trabajo
            \item Pronósticos de costos
            \item Solicitudes de cambio
            \item Actualizaciones al plan para la dirección del proyecto
            \item Actualizaciones a los documentos del proyecto (estimaciones, cronograma, riesgos)
            \item Registro de lecciones aprendidas
        \end{itemize}

        \textbf{Indicadores Clave}
        \begin{itemize}
            \item \textbf{EV (Valor Ganado):} Trabajo completado expresado en términos de presupuesto autorizado.
            \item \textbf{AC (Costo Real):} Costo incurrido por el trabajo realizado.
            \item \textbf{PV (Valor Planificado):} Presupuesto asignado al trabajo planificado.
            \item \textbf{CPI:} Eficiencia del costo. $CPI = \frac{EV}{AC}$
            \item \textbf{SPI:} Eficiencia del cronograma. $SPI = \frac{EV}{PV}$
        \end{itemize}
      \end{multicols}

    \subsubsection{Gestión de la Calidad del Proyecto}
      \begin{multicols}{2}
        \textbf{Objetivo}
Asegurar que los entregables del proyecto GIS cumplan con los requisitos técnicos, funcionales y de interoperabilidad definidos, garantizando la satisfacción del cliente y el cumplimiento de estándares internacionales de calidad.

\textbf{Procesos Clave}
\begin{itemize}
    \item \textbf{Planificar la Gestión de la Calidad:} Identificar los estándares de calidad aplicables al proyecto y documentar cómo se demostrará su cumplimiento.
    \item \textbf{Gestionar la Calidad:} Implementar actividades de mejora continua, revisión de procesos y validación técnica durante la ejecución del proyecto.
    \item \textbf{Controlar la Calidad:} Monitorear los entregables y procesos para verificar que cumplan con los requisitos establecidos, aplicando métricas y auditorías.
\end{itemize}

\textbf{Actividades de Calidad}
\begin{itemize}
    \item Validación de entregables principales por el comité de dirección.
    \item Revisión de pares del diseño del aplicativo por analistas expertos.
    \item Incorporación de controles de calidad en todos los procesos organizacionales.
    \item Auditoría del proyecto al cierre de la planificación.
\end{itemize}

\textbf{Métricas de Calidad}
\begin{itemize}
    \item Máximo de 40 errores por ciclo de prueba (supervisor de pruebas).
    \item Tiempo de respuesta inferior a 2 segundos por petición (supervisor técnico).
    \item Disponibilidad mínima del 95\% durante el primer año de uso.
    \item Nivel de satisfacción del cliente superior al 90\% en encuestas de cierre.
\end{itemize}

\textbf{Estándares y Referencias}
\begin{itemize}
    \item ISO 9000 – Definición de calidad como cumplimiento de requisitos.
    \item ISO 19115 y OGC – Estándares internacionales de geoinformación.
    \item Principios de Juran, Crosby y Deming sobre mejora continua y prevención.
\end{itemize}

\textbf{Consideraciones para Entornos Adaptativos}
\begin{itemize}
    \item Incorporación de revisiones frecuentes de calidad durante el ciclo de vida.
    \item Retrospectivas periódicas para identificar causas raíz y proponer mejoras.
    \item Validación temprana de entregables en lotes pequeños para reducir costos de corrección.
\end{itemize}

\textbf{Entradas, Herramientas y Salidas}
\textbf{Entradas:}
\begin{itemize}
    \item Acta de constitución del proyecto
    \item Plan para la dirección del proyecto
    \item Documentación de requisitos y matriz de trazabilidad
    \item Factores ambientales y activos de procesos organizacionales
\end{itemize}

\textbf{Herramientas y Técnicas:}
\begin{itemize}
    \item Juicio de expertos
    \item Análisis costo-beneficio y análisis de datos
    \item Reuniones, entrevistas y tormenta de ideas
    \item Diagramas de flujo, métricas de calidad, planificación de pruebas
\end{itemize}

\textbf{Salidas:}
\begin{itemize}
    \item Plan de gestión de la calidad
    \item Métricas de calidad
    \item Actualizaciones al plan para la dirección del proyecto
    \item Actualizaciones a los documentos del proyecto
    \item Registro de lecciones aprendidas
\end{itemize}

\textbf{Responsabilidades:}
\begin{itemize}
    \item El equipo de proyecto es responsable de la implementación de las actividades de gestión de la calidad.
    \item El líder de proyecto debe asegurar que se asignen los recursos necesarios para cumplir con los estándares de calidad.
    \item El comité de dirección debe validar y aprobar los entregables del proyecto.
\end{itemize}
      \end{multicols}

      \subsubsection{Gestión de los Recursos del Proyecto}

      \begin{multicols}{2}

\textbf{Objetivo}
Identificar, adquirir y gestionar los recursos humanos, técnicos y físicos necesarios para la ejecución exitosa del proyecto GIS, asegurando disponibilidad, eficiencia operativa y alineación con los objetivos estratégicos.

\textbf{Procesos Clave}
\begin{itemize}
    \item \textbf{Planificar la Gestión de Recursos:} Definir cómo se estimarán, asignarán y gestionarán los recursos del proyecto.
    \item \textbf{Adquirir Recursos:} Obtener los recursos físicos y humanos necesarios para ejecutar el trabajo del proyecto.
    \item \textbf{Desarrollar el Equipo:} Mejorar las competencias, interacción y ambiente del equipo para maximizar el desempeño.
    \item \textbf{Dirigir el Equipo:} Supervisar el desempeño, resolver conflictos y asegurar la motivación y compromiso del equipo.
    \item \textbf{Controlar los Recursos:} Asegurar que los recursos asignados se utilicen según lo planificado y gestionar ajustes cuando sea necesario.
\end{itemize}

\textbf{Tipos de Recursos}
\begin{itemize}
    \item \textbf{Humanos:} Gerente de Proyecto, Especialistas GIS, PMO, Analistas Técnicos, Capacitadores.
    \item \textbf{Tecnológicos:} Plataformas GIS (WebGIS, Desktop), servidores, software de análisis espacial.
    \item \textbf{Materiales:} Manuales impresos, equipos de capacitación, licencias de software (cuando aplican).
\end{itemize}

\textbf{Herramientas y Técnicas}
\begin{itemize}
    \item Juicio de expertos
    \item Asignación de roles y responsabilidades (matriz RACI)
    \item Evaluaciones de desempeño
    \item Reuniones de seguimiento y coaching
    \item Software de gestión de recursos y cronograma
\end{itemize}

\textbf{Salidas}
\begin{itemize}
    \item Registro de recursos
    \item Calendario de recursos
    \item Evaluaciones de desempeño
    \item Actualizaciones al plan para la dirección del proyecto
    \item Registro de lecciones aprendidas
\end{itemize}

\textbf{Consideraciones Estratégicas}
\begin{itemize}
    \item Asignación clara de roles para evitar saturación del equipo técnico.
    \item Capacitación continua para asegurar la calidad de los entregables.
    \item Gestión colaborativa para fomentar la innovación y el compromiso.
    \item Uso eficiente de recursos tecnológicos para garantizar interoperabilidad.
\end{itemize}

      \end{multicols}
\newpage
      \subsubsection{Gestión de las Comunicaciones del Proyecto}

      \begin{figure}[H]
        \centering
        \includegraphics[width=1.3\textwidth,angle=90,trim=1cm 1cm 1cm 1cm,clip]{assets/comunicaciones.png}
        \caption{Diagrama de flujo de las comunicaciones del proyecto.}
        \label{fig:comunicaciones}
      \end{figure}

  \newpage

  \subsubsection{Gestión de los Riesgos del Proyecto}
  \begin{figure}[H]
        \centering
        \includegraphics[width=1.3\textwidth,angle=90,trim=1cm 1cm 1cm 1cm,clip]{assets/riesgos.png}
        \caption{Diagrama de flujo de los riesgos del proyecto.}
        \label{fig:riesgos}
      \end{figure}

  \subsubsection{Gestión de las Adquisiciones del Proyecto}

\begin{multicols}{2}
  \textbf{Objetivo}
Gestionar de forma eficiente la compra de bienes, servicios y recursos externos necesarios para la ejecución del proyecto GIS, asegurando cumplimiento contractual, calidad técnica y optimización de costos.

\textbf{Procesos Clave}
\begin{itemize}
    \item \textbf{Planificar la Gestión de las Adquisiciones:} Identificar qué recursos deben ser adquiridos externamente, definir criterios de selección y establecer el enfoque contractual.
    \item \textbf{Efectuar las Adquisiciones:} Solicitar cotizaciones, evaluar proveedores, negociar contratos y formalizar acuerdos.
    \item \textbf{Controlar las Adquisiciones:} Supervisar el cumplimiento de los contratos, gestionar entregas, validar calidad y resolver incidencias.
\end{itemize}

\textbf{Bienes y Servicios Adquiribles}
\begin{itemize}
    \item Licencias de software GIS (cuando no se utilicen soluciones open source).
    \item Servicios de hosting y servidores para la plataforma WebGIS.
    \item Equipos de capacitación (proyectores, manuales impresos, estaciones de trabajo).
    \item Servicios especializados en análisis geoespacial o interoperabilidad (si no están disponibles internamente).
\end{itemize}

\textbf{Entradas}
\begin{itemize}
    \item Acta de constitución del proyecto
    \item Plan para la dirección del proyecto
    \item Requisitos técnicos y funcionales
    \item Presupuesto aprobado
    \item Políticas de compras de la organización
\end{itemize}

\textbf{Herramientas y Técnicas}
\begin{itemize}
    \item Juicio de expertos
    \item Análisis de mercado y proveedores
    \item Solicitud de propuestas (RFP) y cotizaciones (RFQ)
    \item Evaluación multicriterio
    \item Negociación contractual
    \item Sistema de información para la gestión de adquisiciones
\end{itemize}

\textbf{Salidas}
\begin{itemize}
    \item Contratos firmados
    \item Registro de adquisiciones
    \item Actualizaciones al plan para la dirección del proyecto
    \item Informes de desempeño de proveedores
    \item Registro de lecciones aprendidas
\end{itemize}

\textbf{Consideraciones Estratégicas}
\begin{itemize}
    \item Priorizar proveedores con experiencia en proyectos GIS y cumplimiento de estándares internacionales.
    \item Asegurar cláusulas de calidad, soporte técnico y propiedad intelectual en los contratos.
    \item Coordinar con el área de Finanzas para garantizar disponibilidad presupuestaria y trazabilidad.
    \item Documentar todo el proceso de adquisición para facilitar auditorías y futuras referencias.
\end{itemize}

\end{multicols}

\section{Conclusiones}

La gestión de adquisiciones en el proyecto GIS es fundamental para asegurar el acceso a los recursos necesarios y optimizar los costos. A través de una planificación adecuada, la selección de proveedores calificados y el control riguroso de los contratos, se puede garantizar el éxito del proyecto y la satisfacción de los interesados. La documentación y el aprendizaje continuo también son clave para mejorar los procesos en futuras iniciativas.

Los procesos de gestión de adquisiciones permiten una administración estructurada y eficiente, minimizando riesgos y asegurando que los recursos adquiridos cumplan con los requisitos técnicos y funcionales del proyecto. La colaboración estrecha con el equipo del proyecto y las áreas involucradas es esencial para alinear las adquisiciones con los objetivos estratégicos y operativos de la organización.

La gestión de adquisiciones no solo contribuye a la eficiencia operativa, sino que también fortalece la capacidad de la organización para llevar a cabo proyectos complejos en el futuro, estableciendo una base sólida para la mejora continua y la innovación en la gestión de proyectos GIS.