\documentclass[12pt,a4paper]{article}
\usepackage[utf8]{inputenc}
\usepackage[spanish]{babel}
\usepackage{amsmath}
\usepackage{amsfonts}
\usepackage{amssymb}
\usepackage{graphicx}
\usepackage{setspace}
\usepackage{hyperref}
\usepackage{bookmark}

\title{Análisis Comparativo de Metodologías de Gestión de Proyectos:\\ Casos de Estudio PMI y Ágil}
\author{[Tu Nombre]}
\date{\today}

\begin{document}
\maketitle

\begin{abstract}
Este documento presenta un análisis comparativo de dos casos reales de implementación de metodologías de gestión de proyectos en empresas locales. Se analiza un caso utilizando la metodología PMI (clásica) y otro caso utilizando metodología Ágil, comparando sus características, beneficios y desafíos.
\end{abstract}

\section{Caso 1: Proyecto con Metodología PMI}

Las metodologías de gestión de proyectos son fundamentales para el éxito de cualquier iniciativa. En este caso, se analizará un proyecto que implementó la metodología PMI (Project Management Institute), la cual se caracteriza por su enfoque estructurado y basado en procesos.

Según: Aspectos Generales del Proyecto 

Nombre de la empresa: Binswanger Perú.

RUC: 20506628963.

Razón Social: Administracion inmobiliaria sociedad anonima cerrada.

Descripción de la Empresa:


Según (Gerencia de Proyectos y Construcción - Binswanger Perú, n.d.) Ofrecemos un servicio integral de Gerencia de Proyectos, respaldado por más de 20 años de experiencia local que asegura la planificación, coordinación y control eficiente de cada etapa de tu proyecto inmobiliario, desde la concepción hasta la entrega final.
Según (Gerencia de Proyectos y Construcción - Binswanger Perú, n.d.) Ofrecemos un servicio integral de Gerencia de Proyectos, respaldado por más de 20 años de experiencia local que asegura la planificación, coordinación y control eficiente de cada etapa de tu proyecto inmobiliario, desde la concepción hasta la entrega final.

Nuestro enfoque en la gestión de calidad, prevención de riesgos, optimización de recursos y coordinación de equipos multidisciplinarios asegura que cada proyecto se ejecute de manera fluida, minimizando imprevistos y maximizando el valor para nuestros clientes. Nos comprometemos con la seguridad, la sostenibilidad y la excelencia operativa en cada obra. 

En ese sentido, según la empresa, su Project Manager es responsable de liderar y coordinar todos los aspectos del proyecto, asegurando que se cumplan los objetivos establecidos y se mantenga la calidad en cada fase del proceso.

Dentro de las habilidades de este project manager podemos destacar:

\begin{itemize}
    \item Liderazgo: Capacidad para guiar y motivar al equipo hacia el cumplimiento de los objetivos del proyecto.
    \item Comunicación: Habilidad para transmitir información de manera clara y efectiva a todos los interesados.
    \item Planificación: Capacidad para desarrollar un plan de proyecto detallado que incluya cronogramas, recursos y presupuestos.
    \item Gestión de Riesgos: Habilidad para identificar, evaluar y mitigar riesgos potenciales que puedan afectar el proyecto.
    \item Resolución de Problemas: Capacidad para abordar y resolver problemas de manera efectiva y eficiente.
\end{itemize}

El ciclo de vida de los proyectos de esta empresa se basa en las fases del ciclo de vida del proyecto definidas por el PMI, que incluyen:

\begin{itemize}
    \item Inicio: Definición del proyecto y obtención de la aprobación.
    \item Planificación: Desarrollo de un plan detallado que guiará la ejecución del proyecto.
    \item Ejecución: Implementación del plan y gestión del equipo del proyecto.
    \item Monitoreo y Control: Seguimiento del progreso y realización de ajustes según sea necesario.
    \item Cierre: Finalización formal del proyecto y entrega de resultados.
\end{itemize}

Algunos beneficios de la aplicación de la metodología PMI en esta empresa incluyen una mayor claridad en la definición de roles y responsabilidades, lo que facilita la comunicación y la colaboración entre los miembros del equipo. Además, la estructura del ciclo de vida del proyecto permite un mejor control de los plazos y los costos, lo que se traduce en una mayor satisfacción del cliente.

Las desventajas que se pueden encontrar, para este caso en concreto pueden ser la rigidez de la metodología, que puede no adaptarse bien a proyectos con requisitos cambiantes, y la posible sobrecarga administrativa que puede generar en proyectos más pequeños.

\subsection{Aspectos Generales del Proyecto}
\subsubsection{Nombre de la empresa y/o Área}
[Nombre de la empresa]

\subsubsection{Descripción de la empresa}
[Descripción detallada de la empresa]

\subsubsection{Misión}
[Misión de la empresa]

\subsubsection{Visión}
[Visión de la empresa]

\subsubsection{Descripción del proyecto}
[Descripción detallada del proyecto]

\subsubsection{Metodología implementada}
[Descripción de la metodología PMI implementada]

\subsection{Características del Modelo PMI}
\subsubsection{Roles y responsabilidades del Project Manager}
[Detallar roles y responsabilidades]

\subsubsection{Habilidades requeridas para un Project Manager}
[Listar y describir habilidades]

\subsubsection{Ciclo de Vida del Proyecto}
[Describir las fases del ciclo de vida]

\subsubsection{Beneficios del uso de la metodología}
[Enumerar y explicar beneficios]

\subsubsection{Desventajas del uso de la metodología}
[Enumerar y explicar desventajas]

\section{Caso 2: Proyecto con Metodología Ágil}
\subsection{Aspectos Generales del Proyecto}
\subsubsection{Nombre de la empresa y/o Área}
[Nombre de la empresa]

\subsubsection{Descripción de la empresa}
[Descripción detallada de la empresa]

\subsubsection{Misión}
[Misión de la empresa]

\subsubsection{Visión}
[Visión de la empresa]

\subsubsection{Descripción del proyecto}
[Descripción detallada del proyecto]

\subsubsection{Metodología implementada}
[Descripción de la metodología Ágil implementada]

\subsection{Características del Modelo Ágil}
\subsubsection{Roles y responsabilidades del Scrum Master}
[Detallar roles y responsabilidades]

\subsubsection{Habilidades requeridas para un Scrum Master}
[Listar y describir habilidades]

\subsubsection{Ciclo de Vida del Proyecto}
[Describir las fases del ciclo de vida]

\subsubsection{Beneficios del uso de la metodología}
[Enumerar y explicar beneficios]

\subsubsection{Desventajas del uso de la metodología}
[Enumerar y explicar desventajas]

\section{Conclusiones}
[Al menos 4 conclusiones basadas en los resultados de los proyectos trabajados]

% \begin{thebibliography}{9}
% % Aquí irán las referencias en formato APA
% \end{thebibliography}

\end{document}
