\section{Enunciado del Alcance del Proyecto}

El producto principal consiste en el diseño y desarrollo de una plataforma integral de servicios y entregables técnicos, que facilite a los clientes la optimización de sus procesos de toma de decisiones mediante el uso de información geoespacial. El desarrollo se realizará conforme a estándares internacionales de calidad y buenas prácticas en gestión de proyectos y GIS.

\subsection{Entregables}

\begin{itemize}
  \item Diagnóstico de necesidades y levantamiento de requerimientos GIS de los clientes.
  \item Base de datos geoespacial estructurada y documentada.
  \item Mapas temáticos digitales y en formato impreso (por ejemplo: uso de suelo, cobertura vegetal, infraestructura, riesgo).
  \item Informes técnicos de análisis espacial (por ejemplo: modelamiento de escenarios, identificación de patrones territoriales, análisis multitemporal).
  \item Plataforma digital para consulta y visualización de información geográfica.
  \item Capacitación inicial a los usuarios finales en el uso de herramientas GIS y en la interpretación de productos cartográficos.
\end{itemize}

\subsection{Entregables de Gestión (según guía PMBOK)}

\begin{itemize}
  \item Acta de Constitución del Proyecto.
  \item Gestión del alcance del proyecto.
  \item Estructura de Desglose del Trabajo (EDT).
  \item Plan de Gestión del Cronograma.
\end{itemize}

\subsection{Criterios de Aceptación}

Los entregables del proyecto serán aceptados si cumplen con los siguientes criterios:

\begin{itemize}
  \item Correspondencia con los requisitos establecidos por los interesados y documentados en la fase de planificación.
  \item Validación de la calidad de los datos espaciales (precisión, consistencia, completitud).
  \item Entrega de productos cartográficos y bases de datos en los formatos solicitados por el cliente.
  \item Aprobación formal por parte del patrocinador del proyecto y de los principales interesados.
\end{itemize}

\subsection{Exclusiones del Proyecto}

Quedan fuera del alcance del presente proyecto las siguientes actividades:

\begin{itemize}
  \item Mantenimiento y actualización continua de la base de datos geoespacial una vez entregada al cliente.
  \item Adquisición de licencias de software propietario (estas serán asumidas directamente por el cliente en caso de requerirse).
  \item Soporte técnico posterior al periodo de capacitación inicial.
  \item Integración de la plataforma con sistemas externos no contemplados en los requisitos iniciales.
\end{itemize}

Estas exclusiones permiten delimitar claramente las responsabilidades del proyecto, evitando desviaciones y gestionando de manera efectiva las expectativas de los interesados.