\section{Acta de Constitución del Proyecto}
\textit{Guía del PMBOK® 4.1.3.1}

\renewcommand{\arraystretch}{1.2}
\setlength{\tabcolsep}{5pt}

\begin{tabular}{|>{\raggedright\arraybackslash}p{4cm}|>{\raggedright\arraybackslash}p{12.2cm}|}
\hline
\multicolumn{1}{|>{\centering\arraybackslash}p{4cm}|}{\textbf{Componente}} & \multicolumn{1}{>{\centering\arraybackslash}p{12.2cm}|}{\textbf{Descripción}} \\
\hline
Equipo & Global GIS Service GGS \\
\hline
Nombre del Proyecto & Consultoría Integral en Soluciones GIS. \\
\hline
Gerente del Proyecto & Luis Huatay – Gerente del Proyecto \\
\hline
Patrocinador del Proyecto & Dra. Rosmery R. – Directora de Investigación y Proyectos Estratégicos \\
\hline
Descripción del Proyecto & Este proyecto consiste en el diseño y ejecución de servicios de consultoría especializados en Sistemas de Información Geográfica (SIG/GIS).\\
\hline
Justificación del Proyecto & Aprovechar los siguientes factores: \newline
- Creciente necesidad de las organizaciones por gestionar información geoespacial. \newline
- Oportunidad de generar ventajas competitivas a través del análisis espacial y la visualización de datos. \newline
- Demanda en sectores público y privado por soluciones que integren datos geográficos con procesos de negocio. \newline
- Potencial de posicionar a la organización como referente en consultoría GIS en el mercado local e internacional. \\
\hline
Objetivos del proyecto y criterios de medición del éxito &
- Brindar servicios de consultoría GIS que satisfagan las necesidades específicas de los clientes (criterio: contratos ejecutados con éxito). \newline
- Entregar propuestas de soluciones GIS personalizadas en un tiempo máximo de 8 semanas por cliente (criterio: cumplimiento de cronograma). \\
\hline
Requerimientos Principales (Alto nivel) & 
- Plataformas GIS. \newline
- Personal especializado en análisis espacial, bases de datos geográficas y programación. \newline
- Capacitación básica para usuarios finales. \newline
- Documentación técnica y reportes ejecutivos. \\
\hline
Riesgos Principales (Alto nivel) & 
- Problemas de interoperabilidad entre sistemas GIS y las plataformas existentes. \newline
- Limitaciones presupuestarias de algunas organizaciones. \newline
- Dependencia de datos externos que pueden no estar disponibles o ser de baja calidad. \\
\hline
Resumen del Cronograma de Hitos & 
- Septiembre - Octubre: Definición y constitución de la empresa. \newline
- Noviembre - Diciembre: Desarrollo de la oferta de servicios y marketing inicial. \\
\hline
Presupuesto Resumido (Orden de Magnitud) & - 20000 en 4 meses \\
\hline
Supuestos & 
- Disponibilidad de datos geoespaciales confiables. \newline
- Colaboración activa por parte de los clientes. \newline
- Acceso a infraestructura tecnológica mínima en cada organización. \\
\hline
Restricciones & 
- Presupuesto limitado a 20000 en 4 meses. \newline
- Plazos ajustados para la entrega de propuestas y servicios. \newline
- Recursos humanos especializados en GIS pueden ser limitados. \\
\hline
Interesados & 
- Ingeniero de Software con especialización en sistemas de información geográfica.
\newline- Directora de Investigación y Proyectos Estratégicos. \\
\hline
Requerimientos de aprobación del proyecto & 
- Unanimidad del equipo directivo y patrocinador del proyecto. \newline
- Aprobación formal del acta de constitución del proyecto. \\
\hline
\end{tabular}
