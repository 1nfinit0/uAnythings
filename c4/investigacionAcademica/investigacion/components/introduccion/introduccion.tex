\section{Matriz de Consistencia}

La siguiente tabla presenta la matriz de consistencia que detalla los elementos clave de la investigación propuesta, incluyen el problema, delimitaciones, preguntas e hipótesis.

\newcolumntype{M}[1]{>{\hspace{0cm}\justifying\arraybackslash}m{#1}<{\hspace{0cm}}}
\newcolumntype{J}[1]{>{\hspace{0cm}\justifying\arraybackslash}m{#1}<{\hspace{0cm}}} % Columna justificada

\begin{table}[h!]
  \centering
  \renewcommand{\arraystretch}{2.5} % Más espacio vertical
  \begin{tabular}{|M{4cm}|M{5cm}|J{6.5cm}|}
    \hline
    \textbf{Problema} & Describe el problema central que abordará tu investigación. Debe ser claro, específico y relevante. & {Uso del modelo AlphaEarth Foundations para clasificación de cultivos del sector agrícola del Perú entre los años 2020 y 2024.} \\
    \hline
    \textbf{Delimitación espacial} & Especifica el lugar o ámbito geográfico donde se realizará la investigación. & {Sector agrícola del Perú} \\
    \hline
    \textbf{Delimitación temporal} & Define el período de tiempo en el que se enfocará la investigación. & {Años 2020 a 2024} \\
    \hline
    \textbf{Delimitación temática} & Especifica el área de interés, sus límites y alcances de lo que se va a estudiar & {Uso del modelo AlphaEarth Foundations para clasificación de cultivos} \\
    \hline
    \textbf{Pregunta} & Plantea preguntas de investigación que guíen tu estudio. Deben ser claras y viables. & {¿Cómo, el uso del modelo AlphaEarth Foundations, puede mejorar la clasificación de cultivos agrícolas en el Perú entre los años 2020 y 2024?} \\
    \hline
    \textbf{Hipótesis} & Propone una respuesta tentativa al problema de investigación. Debe ser comprobable. & {El modelo AlphaEarth Foundations mejora la clasificación de cultivos en el sector agrícola del Perú entre los años 2020 y 2024 gracias a su capacidad para integrar datos geoespaciales y de sensores remotos impulsados por inteligencia artificial.} \\
    \hline
  \end{tabular}
  \caption{Matriz de Consistencia}
  \label{tab:matriz_consistencia}
\end{table}