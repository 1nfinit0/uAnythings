\section{Justificación}

La justificacion del proyecto radica en la necesidad de optimizar la clasificación de cultivos en el sector agrícola del Perú.

\subsection{¿Cuáles son nuestros intereses que nos llevan a escoger este tema?}

Nuestros intereses se centran en la aplicación de tecnologías avanzadas, como el modelo AlphaEarth Foundations, para mejorar la eficiencia y precisión en la clasificación de cultivos en un mundo donde cada vez más es solicitada las soluciones basadas en los sistemas de información geográfica (SIG) y la inteligencia artificial (IA). La herramienta que recientemente ha sido lanzada para el uso público, nos brinda la oportunidad de explorar su potencial para resolver problemas reales en el sector agrícola.

\subsection{¿Cuál es la relevancia académica del tema propuesto?}

Esta investigación es relevante porque aporta evidencia empírica sobre la efectividad del modelo AlphaEarth Foundations en la clasificación de cultivos, lo que puede contribuir al avance del conocimiento, toma de decisiones informadas por parte de agricultores y responsables de políticas, además de fomentar el uso y desarrollo de tecnologías similares en el futuro.

\section{Referencias Bibliográficas}

\subsection{Tésis universitarias}

\textbf{Simulación de la función de reflectancia para la resonancia de plasmones de superficie. \cite{Superficie2024}}

\begin{itemize}
  \item \textbf{Autores:} Orna Tiburcio, Diego André, Asesor: Bruna Mercado.
  \item \textbf{Año:} 2024.
  \item \textbf{Título:} Simulación de la función de reflectancia para la resonancia de plasmones de superficie.
  \item \textbf{Ciudad:} Lima, Perú.
  \item \textbf{Universidad:} Universidad Nacional Federico Villarreal.
  \item \textbf{Tipo de tesis:} Tesis para obtener la licenciatura en Física.
  \item \textbf{Enlace:} \url{https://repositorio.unfv.edu.pe/handle/20.500.13084/9880}
  \item \textbf{Pertinencia:}
  \subitem \textbf{Autoridad:} Se trata de una tesis de licenciatura para obtener el título profesional de Físico, respaldada por la Facultad de Ciencias Naturales y Matemática de la Universidad Nacional Federico Villarreal.
  \subitem \textbf{Contenido:} El trabajo se centra en el estudio y la simulación computacional de la resonancia de plasmones de superficie (SPR) bajo la configuración de Kretschmann.
  \subitem \textbf{Objetividad:} Desarrollar un programa en Wolfram Mathematica para modelar y comprender el comportamiento de la función de reflectancia al variar parámetros experimentales clave como el espesor de la película metálica, la longitud de onda, la función dieléctrica del metal y los índices de refracción del prisma y el dieléctrico.
  \subitem \textbf{Extensión:} La tesis aborda el tema de manera exhaustiva dentro de su alcance.
\end{itemize}

\textbf{Principios de teledetección y sus aplicaciones. \cite{En2009}}
\begin{itemize}
  \item \textbf{Autores:} Marcelo Geomensura, R Beltrán, B Jorge, D Araneda.
  \item \textbf{Año:} 2009.
  \item \textbf{Título:} Principios de teledetección y sus aplicaciones.
  \item \textbf{Ciudad:} Concepción, Chile.
  \item \textbf{Universidad:} Universidad de Concepción.
  \item \textbf{Tipo de tesis:} Tesis presentada para optar al título de Ingeniero de Ejecución en Geomensura.
  \item \textbf{Enlace:} \url{https://repositorio.udec.cl/items/a9077392-2db5-4186-b309-1c1d4b6137dd/}
  \item \textbf{Pertinencia:}
  \subitem \textbf{Autoridad:} Es un trabajo de titulación (Informe de Habilitación Profesional) para obtener un grado de ingeniería, respaldado por la Universidad de Concepción.
  \subitem \textbf{Contenido:} El informe es un trabajo de investigación que expone los fundamentos teóricos y las aplicaciones de la teledetección (percepción remota), con un enfoque especial en su utilidad para el área de la Geomensura.
  \subitem \textbf{Objetividad:} Proporcionar una visión integral de las técnicas y tecnologías utilizadas en la teledetección, así como ejemplos prácticos de su aplicación en estudios geográficos y ambientales.
  \subitem \textbf{Extensión:} El tratamiento del tema es exhaustivo. La investigación cubre los fundamentos físicos, los componentes tecnológicos, diversas aplicaciones prácticas con proyectos específicos.
\end{itemize}

\subsection{Artículos académicos}

\textbf{AlphaEarth Foundations: An embedding field
model for accurate and efficient global
mapping from sparse label data. \cite{Brown2025}}

\begin{itemize}
  \item \textbf{Base de datos:} arxiv.org
  \item \textbf{Autores:}  Brown, Christopher F.; Kazmierski, Michal R.; Pasquarella, Valerie J.; Rucklidge, William J.; Samsikova, Masha; Zhang, Chenhui; Shelhamer, Evan; Lahera, Estefania; Wiles, Olivia; Ilyushchenko, Simon; Gorelick, Noel; Zhang, Lihui Lydia; Alj, Sophia; Schechter, Emily; Askay, Sean; Guinan, Oliver; Moore, Rebecca; Boukouvalas, Alexis; y Kohli, Pushmeet.
  \item \textbf{Título:} AlphaEarth Foundations: An embedding field model for accurate and efficient global mapping from sparse label data.
  \item \textbf{Nombre de la Revista:} Preprint en arxiv.org
  \item \textbf{N° de Revista:} Preprint aún en revisión.
  \item \textbf{Enlace:} \url{https://arxiv.org/abs/2507.22291}
  \item \textbf{Pertinencia:}
  \subitem \textbf{Autoridad:} El artículo es publicado por DeepMind y revisado por pares en la prestigiosa revista Nature.
  \subitem \textbf{Contenido:} Presenta un modelo llamado AlphaEarth Foundations que utiliza datos de etiquetas escasas para crear mapas globales precisos y eficientes.
  \subitem \textbf{Objetividad:} El artículo describe detalladamente la arquitectura del modelo, los métodos de entrenamiento y los resultados experimentales que demuestran su eficacia en comparación con enfoques previos.
  \subitem \textbf{Extensión:} El artículo es exhaustivo y cubre todos los aspectos relevantes del modelo y sus aplicaciones potenciales en mapeo geoespacial.
\end{itemize}

\textbf{Precision agriculture technologies for soil site-specific nutrient
management: A comprehensive review. \cite{Vullaganti2025}}

\begin{itemize}
  \item \textbf{Base de datos:} ScienceDirect
  \item \textbf{Autores:} Vullaganti, Anil Kumar; Singh, Rajendra; Singh, Satyendra; y Kumar, Santosh.
  \item \textbf{Título:} Precision agriculture technologies for soil site-specific nutrient management: A comprehensive review.
  \item \textbf{Nombre de la Revista:} Computers and Electronics in Agriculture.
  \item \textbf{N° de Revista:} Volume 220, 2025, 107231.
  \item \textbf{Enlace:} \url{https://www.sciencedirect.com/science/article/pii/S2589721725000200}
  \item \textbf{Pertinencia:}
  \subitem \textbf{Autoridad:} El artículo es publicado en la revista Computers and Electronics in Agriculture, una fuente confiable y revisada por pares en el campo de la agricultura de precisión.
  \subitem \textbf{Contenido:} Proporciona una revisión exhaustiva de las tecnologías utilizadas en la agricultura de precisión para la gestión específica de nutrientes en el suelo.
  \subitem \textbf{Objetividad:} El artículo analiza diversas tecnologías, sus aplicaciones y beneficios, así como los desafíos asociados con su implementación en la agricultura moderna.
  \subitem \textbf{Extensión:} La revisión es completa y abarca una amplia gama de tecnologías y enfoques utilizados en la gestión de nutrientes del suelo en la agricultura de precisión.
\end{itemize}

\textbf{Principles, developments,
and applications of spatially
resolved spectroscopy in
agriculture: a review. \cite{Xia2023}}
\begin{itemize}
  \item \textbf{Base de datos:} Frontiers.
  \item \textbf{Autores:} Xia, J.; Zhang, L.; y He, Y.
  \item \textbf{Título:} Principles, developments, and applications of spatially resolved spectroscopy in agriculture: a review.
  \item \textbf{Nombre de la Revista:} Frontiers in Plant Science.
  \item \textbf{N° de Revista:} Volume 14, (2023).
  \item \textbf{Enlace:}
  
  \url{www.frontiersin.org/journals/plant-science/articles/10.3389/fpls.2023.1324881/full}
  \item \textbf{Pertinencia:}
  \subitem \textbf{Autoridad:} El artículo es publicado en la revista Frontiers in Plant Science, una fuente confiable y revisada por pares en el campo de la agricultura de precisión.
  \subitem \textbf{Contenido:} Proporciona una revisión exhaustiva de los principios, desarrollos y aplicaciones de la espectroscopía espacialmente resuelta en la agricultura.
  \subitem \textbf{Objetividad:} El artículo analiza diversas técnicas espectroscópicas, sus aplicaciones en la agricultura y los beneficios que ofrecen para mejorar la eficiencia y sostenibilidad agrícola.
  \subitem \textbf{Extensión:} La revisión es completa y abarca una amplia gama de técnicas y enfoques utilizados en la espectroscopía espacialmente resuelta en la agricultura.
\end{itemize}

\newpage

\textbf{A systematic review of hyperspectral imaging in precision agriculture:
Analysis of its current state and future prospects. \cite{Ram2024}}

\begin{itemize}
  \item \textbf{Base de datos:} ScienceDirect
  \item \textbf{Autores:} Ram, S.; Kumar, A.; y Singh, R.
  \item \textbf{Título:} A systematic review of hyperspectral imaging in precision agriculture: Analysis of its current state and future prospects.
  \item \textbf{Nombre de la Revista:} Computers and Electronics in Agriculture.
  \item \textbf{N° de Revista:} Volume 220, 2024, 107230.
  \item \textbf{Enlace:} \url{https://www.sciencedirect.com/science/article/pii/S0168169924004289}
  \item \textbf{Pertinencia:}
  \subitem \textbf{Autoridad:} El artículo es publicado en la revista Computers and Electronics in Agriculture, una fuente confiable y revisada por pares en el campo de la agricultura de precisión.
  \subitem \textbf{Contenido:} Proporciona una revisión sistemática del uso de la imagen hiperespectral en la agricultura de precisión, analizando su estado actual y perspectivas futuras.
  \subitem \textbf{Objetividad:} El artículo examina diversas aplicaciones de la imagen hiperespectral en la agricultura, sus beneficios y desafíos asociados con su implementación.
  \subitem \textbf{Extensión:} La revisión es completa en aplicaciones y enfoques relacionados con la imagen hiperespectral en la agricultura de precisión.
\end{itemize}

\subsection{Libros}

\textbf{Remote Sensing Handbook, Volume III: Agriculture, Food Security, Rangelands, Vegetation Phenology, and Soils. \cite{Prasad2025}}
\begin{itemize}
   \item \textbf{Autor:} Prasad S. Thenkabail.
   \item \textbf{Año:} 2024.
  \item \textbf{Título:} Remote Sensing Handbook, Volume III: Agriculture, Food Security, Rangelands, Vegetation Phenology, and Soils.
  \item \textbf{Editorial:} Routledge.
  \item \textbf{Ciudad de publicación:} New York, USA.
  \item \textbf{Enlace:} \href{https://www.routledge.com/Remote-Sensing-Handbook--Volume-III-Agriculture-Food-Security-Rangelands-Vegetation-Phenology-and-Soils/Thenkabail/p/book/9781032891019?srsltid=AfmBOorCvB0BBGEuWlFwMJl1JwtfQgzrvdxRbCWGBck_ocSwrZ9dLpRL}{Remote Sensing Handbook (Routledge)}
  \item \textbf{Pertinencia:}
  \subitem \textbf{Autoridad:} El libro es escrito por Prasad S. Thenkabail, un experto reconocido en el campo de la teledetección y sus aplicaciones en la agricultura.
  \subitem \textbf{Contenido:} El libro aborda en profundidad los principios y aplicaciones de la teledetección en la agricultura, la seguridad alimentaria, los pastizales, la fenología de la vegetación y los suelos.
  \subitem \textbf{Objetividad:} Proporciona una visión integral y actualizada de las tecnologías y metodologías utilizadas en la teledetección para mejorar la gestión agrícola y la seguridad alimentaria.
  \subitem \textbf{Extensión:} El libro es exhaustivo y cubre áreas afines.
\end{itemize}

\newpage

\textbf{Remote Sensing and Image Interpretation, 7th Edition. \cite{Lillesand2015}}

\begin{itemize}
   \item \textbf{Autores:} Thomas M. Lillesand, Ralph W. Kiefer, Jonathan W. Chipman.
   \item \textbf{Año:} 2015.
  \item \textbf{Título:} Remote Sensing and Image Interpretation, 7th Edition.
  \item \textbf{Editorial:} Wiley.
  \item \textbf{Ciudad de publicación:} Hoboken, New Jersey, USA.
  \item \textbf{Enlace:} \href{https://www.wiley.com/en-us/Remote+Sensing+and+Image+Interpretation%2C+7th+Edition-p-9781118343289}{Remote Sensing and Image Interpretation (Wiley)}
  \item \textbf{Pertinencia:}
  \subitem \textbf{Autoridad:} El libro es escrito por Thomas M. Lillesand, Ralph W. Kiefer y Jonathan W. Chipman, expertos reconocidos en el campo de la teledetección y la interpretación de imágenes.
  \subitem \textbf{Contenido:} El libro aborda en profundidad los principios y técnicas de la teledetección y la interpretación de imágenes, con aplicaciones en diversas áreas, incluida la agricultura.
  \subitem \textbf{Objetividad:} Proporciona una visión integral y actualizada de las tecnologías y metodologías utilizadas en la teledetección para mejorar la gestión agrícola y otros campos relacionados.
  \subitem \textbf{Extensión:} El libro es exhaustivo y cubre áreas afines.
\end{itemize}
