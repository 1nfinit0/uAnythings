\section{Informe sobre la generación del PITCH}
\label{sec:introduccion}
\subsection{Introducción}
El presente informe detalla el proceso de creación del PITCH, un documento esencial para presentar ideas de negocio de manera concisa y efectiva. A lo largo de este informe, se describen las etapas de investigación, desarrollo y refinamiento que llevaron a la elaboración del PITCH final. Se aborda la realización de este bajo indicaciones brindadas por la retroalimentación recibida en clase, así como la incorporación de elementos clave para captar la atención de potenciales inversores o socios.

\subsection{Objetivo del PITCH}

El objetivo principal del PITCH es comunicar de manera clara y persuasiva la propuesta de valor de un proyecto o idea de negocio. Este documento busca captar el interés de los lectores, proporcionando una visión general del producto o servicio, el mercado objetivo, la estrategia de negocio y los beneficios esperados. A través de un formato estructurado y atractivo, el PITCH pretende facilitar la comprensión rápida y efectiva de la propuesta, incentivando a los interesados a profundizar en el proyecto.

\subsection{Estructura del PITCH}
El PITCH se estructura en varias secciones clave que permiten una presentación ordenada y coherente de la idea de negocio. A continuación, se describen las principales secciones que componen el PITCH:

\begin{itemize}
  \item \textbf{Storytelling:} Una narrativa que contextualiza la idea de negocio, destacando el problema que se busca resolver y la oportunidad que representa.
  \item \textbf{Propuesta de valor:} Una descripción clara y concisa del producto o servicio ofrecido, enfatizando los beneficios y ventajas competitivas.
  \item \textbf{Análisis de mercado:} Información sobre el mercado objetivo, incluyendo tamaño, segmentación y tendencias relevantes.
  \item \textbf{Modelo de negocio:} Explicación de cómo se generarán ingresos y se sostendrá el negocio a largo plazo.
  \item \textbf{Estrategia de marketing y ventas:} Planes para promocionar el producto o servicio y alcanzar a los clientes potenciales.
  \item \textbf{Equipo:} Presentación del equipo detrás del proyecto, destacando sus habilidades y experiencia relevante.
  \item \textbf{Proyecciones financieras:} Resumen de las expectativas financieras, incluyendo ingresos, costos y rentabilidad.
  \item \textbf{Llamado a la acción:} Una invitación clara para que los lectores tomen el siguiente paso, ya sea invertir, colaborar o solicitar más información.
\end{itemize}

\subsection{Metodología de desarrollo}

Para el desarrollo de este se hará uso de tecnologías como \LaTeX, que permite una presentación profesional y estructurada del documento. Generación de imágenes y gráficos por medio de herramientas de inteligencia artificial como DALL-E y MidJourney, que facilitan la creación de elementos visuales atractivos y relevantes para complementar el contenido del PITCH. Además, se emplearán técnicas de diseño gráfico para asegurar que el documento sea visualmente atractivo y fácil de leer.

\section{Pitch}

\subsection{Storytelling: Un Aula Transformada}

\textbf{Acto I: El Problema}

Imaginemos una escuela secundaria donde, cada semana, los profesores enfrentan la frustración de encontrar plumones de pizarra agotados, computadoras sin mantenimiento y recursos desperdiciados. Las clases se interrumpen, el tiempo se pierde y los estudiantes se desconectan. Esta situación, repetida en miles de aulas, limita el potencial educativo y genera costos innecesarios.

\textbf{Acto II: El Cambio}

En 2023, la Escuela “Innovar” decidió apostar por la tecnología. Implementaron nuestro sistema de automatización: sensores inteligentes monitorean el uso de plumones, proyectores y computadoras; alertas automáticas notifican a los responsables cuando es necesario reponer o mantener recursos. Los profesores ahora se enfocan en enseñar, los recursos se optimizan y los estudiantes disfrutan de un ambiente moderno y eficiente.

\textbf{Acto III: El Éxito}

A los seis meses, la escuela redujo en un 30\% el gasto en consumibles, mejoró la satisfacción docente y aumentó la continuidad de las clases. El éxito fue tal que otras instituciones solicitaron replicar el modelo, marcando el inicio de una revolución en la gestión educativa.

\subsection{Propuesta de Valor}

Ofrecemos una solución integral de automatización para aulas, basada en sensores y circuitos integrados, que permite monitorear y gestionar en tiempo real los recursos clave: plumones, proyectores, computadoras y materiales didácticos. Nuestra plataforma reduce desperdicios, optimiza costos y mejora la experiencia educativa, liberando a los docentes de tareas administrativas y permitiéndoles centrarse en lo más importante: enseñar.

\subsection{Análisis de Mercado}

El mercado objetivo son instituciones educativas públicas y privadas, desde primaria hasta universidades, en América Latina. Existen más de 500,000 escuelas en la región, muchas de las cuales enfrentan problemas de gestión de recursos. El gasto anual en insumos y mantenimiento supera los 2,000 millones de dólares, con una tendencia creciente hacia la digitalización y automatización de procesos educativos.

\subsection{Modelo de Negocio}

Nuestro modelo es B2B, con venta e instalación de kits de automatización y una suscripción mensual por acceso a la plataforma de monitoreo y reportes. Ofrecemos paquetes escalables según el tamaño de la institución, con soporte técnico y actualizaciones incluidas.

\subsection{Estrategia de Marketing y Ventas}

Iniciaremos con pilotos en escuelas emblemáticas y generaremos casos de éxito documentados. Participaremos en ferias educativas, colaboraremos con gobiernos locales y utilizaremos marketing digital dirigido a directores y administradores escolares. Además, estableceremos alianzas con distribuidores de material escolar y tecnología educativa.

\subsection{Equipo Profesional}

Nuestro equipo combina experiencia en ingeniería electrónica, desarrollo de software y gestión educativa. Contamos con especialistas en IoT, diseñadores de hardware y expertos en implementación de soluciones tecnológicas en entornos escolares, respaldados por asesores del sector educativo.

\subsection{Proyecciones Financieras}

Esperamos instalar 100 sistemas en el primer año, generando ingresos por 150,000 USD y alcanzando el punto de equilibrio en 18 meses. Con una tasa de retención del 90\% y expansión regional, proyectamos un crecimiento anual del 60\% y una rentabilidad sostenida a partir del segundo año.

\subsection{Llamado a la Acción}

Invitamos a los inversores a ser parte de la transformación educativa. Con su apoyo, llevaremos la automatización inteligente a miles de aulas, mejorando la eficiencia, reduciendo costos y potenciando el aprendizaje. Juntos, podemos construir el futuro de la educación.