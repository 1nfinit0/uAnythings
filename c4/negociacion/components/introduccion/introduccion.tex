\section{Según las características de la historia a contar}
Nicolás es un profesor que no se deja impresionar por soluciones superficiales. Su objetivo principal es impartir clases eficaces, sin interrupciones, y aprovechar al máximo su tiempo para enseñar. Su mayor preocupación es la pérdida de productividad causada por problemas pequeños pero constantes, como quedarse sin tinta en medio de la clase, tener que caminar largas distancias para solicitar un repuesto o depender de trámites lentos que interrumpen su labor. En su entorno, convive con colegas que enfrentan los mismos problemas, y aunque surgen nuevas propuestas, muchas no logran resolver de manera real y sostenible estas necesidades diarias. Por lo tanto, Nicolás es escéptico frente a cualquier solución que no le muestre beneficios claros, sencillos y comprobables.

Cuando le presentemos el plumón inteligente, Nicolás no querrá escuchar únicamente sobre sus características técnicas; él buscará tranquilidad y eficacia. Querrá saber cómo esta herramienta le garantiza nunca más quedarse sin tinta durante una clase, cómo optimiza su tiempo evitando gestiones innecesarias y cómo además le ayuda a reducir gastos y contribuir al cuidado del medio ambiente. Nuestro principal reto será convencerlo de que, aunque el problema parezca pequeño, el impacto en su experiencia docente es enorme, y que nuestra solución es única, práctica y diseñada para personas como él.

Para captar su atención y persuadirlo, nuestra historia debe centrarse en los beneficios tangibles que vivirá en su día a día: menos frustraciones, mayor eficiencia y un entorno de enseñanza más fluido. No le contaremos una promesa vacía; le contaremos una solución concreta a un problema real.

El mensaje clave de nuestra narrativa será claro: este plumón inteligente con software de recarga automática no es solo un instrumento de escritura, es una herramienta estratégica que devuelve tiempo, reduce gastos y asegura eficacia en el aula. Nuestra historia debe demostrar cómo, al adoptarlo, Nicolás se convertirá en parte de una solución que no solo resuelve una frustración cotidiana, sino que además transforma su experiencia de enseñanza en algo más eficiente, sostenible y libre de preocupaciones.

\newpage

\section{PITCH}

\subsection*{Introducción (Hook)}
Cada año, miles de docentes en Latinoamérica pierden horas valiosas de clase por un problema tan simple como quedarse sin tinta en sus plumones. Lo que parece un detalle menor genera interrupciones, frustración y pérdida de productividad en el aula. Hoy les presentamos una solución única: el Plumón Inteligente con software de recarga automática, diseñado para que ningún profesor vuelva a quedarse sin tinta en medio de una clase.

\subsection*{El Problema}
En universidades y colegios, el 80\% de los docentes usan plumones para dictar clases. Estudios internos muestran que el 65\% de los profesores se ha quedado sin tinta en clase al menos una vez por semana, generando pérdida de tiempo, improvisación y mala experiencia de enseñanza. El proceso actual de solicitar recargas es ineficiente: requiere traslados, tiempo administrativo y genera gastos adicionales en plumones desechables.

\subsection*{La Solución}
Plumón Inteligente con cartucho recargable. Software conectado que anticipa el consumo de tinta y solicita automáticamente la recarga al área de servicio antes de que se agote.

\textbf{Beneficios clave:}
\begin{itemize}
  \item Cero interrupciones en clase.
  \item Ahorro de hasta un 40\% en gastos de plumones al año.
  \item Reducción de residuos plásticos, contribuyendo a la sostenibilidad.
\end{itemize}

\subsection*{Mercado Potencial}
En Latinoamérica hay más de 12 millones de docentes entre colegios y universidades (fuente: UNESCO). Mercado inicial objetivo: docentes universitarios y colegios privados en Perú, estimado en 250,000 usuarios. Expansión regional: con una adopción del 5\% del mercado latinoamericano, el potencial supera los 600,000 usuarios activos en los primeros 5 años.

\subsection*{Modelo de Negocio}
Venta del plumón inteligente (hardware inicial). Suscripción anual al software de gestión y logística de recargas (modelo SaaS). Convenios B2B con universidades y colegios privados para gestión masiva de plumones.

\subsection*{Ventaja Competitiva}
Primera solución en el mercado que integra hardware + software para un problema docente cotidiano. Combina ahorro, sostenibilidad y eficiencia. Escalable: el mismo modelo puede aplicarse a otros útiles escolares y de oficina (marcadores, bolígrafos, etc.).

\subsection*{Tracción y Validación}
Pruebas piloto en universidades de Lima con 30 docentes:
\begin{itemize}
  \item 90\% reportó reducción de frustración en clases.
  \item 80\% destacó el ahorro de tiempo en gestiones administrativas.
\end{itemize}
Interés preliminar de 2 universidades privadas en implementar la solución a escala.

\subsection*{Proyecciones Financieras}
Inversión inicial: USD 300,000 (I+D, producción piloto, marketing).

\textbf{Ingresos proyectados:}
\begin{itemize}
  \item Año 1: USD 150,000 (piloto y primeras licencias).
  \item Año 3: USD 1.2M (expansión nacional y SaaS consolidado).
  \item Año 5: USD 5M (expansión regional con 5\% de penetración).
\end{itemize}
Margen bruto esperado: 60\% en software / 35\% en hardware.

\subsection*{Equipo}
Fundadores con experiencia en educación y tecnología. Equipo multidisciplinario: ingeniería de hardware, desarrollo SaaS y operaciones educativas. Asesoría de especialistas en logística universitaria.

\subsection*{Cierre – El Mensaje Clave}
El plumón inteligente no es solo un instrumento de escritura. Es una solución escalable y sostenible que resuelve un problema real en el sector educativo, ahorra tiempo y dinero a las instituciones, y abre un mercado millonario en Latinoamérica.

Con su inversión, podremos escalar este producto, impactar la experiencia docente de millones de profesores y al mismo tiempo generar retornos financieros sólidos.

Nuestra propuesta es simple: nunca más un docente se quedará sin tinta en medio de una clase.